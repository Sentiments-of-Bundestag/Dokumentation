\documentclass[a4paper,12pt,twoside]{book}

% standard incantations
\usepackage[T1]{fontenc}
\usepackage[utf8]{inputenc}
\usepackage{lmodern}
\usepackage[german]{babel}
\usepackage{csquotes}

% clickable links in the PDF
\usepackage{hyperref}
\usepackage{float}
% bibliography
\usepackage{biblatex}
\addbibresource{literatur.bib}

% glossary
\usepackage[xindy]{glossaries} 
\newglossaryentry{souveraenitaet}{%
  name={Souveränität},%
  description={Der Begriff Souveränität, auch „Staatshoheit“, wird im innerstaatlichen Recht und in der politischen Theorie verwendet, um die oberste Kompetenz zur Machtausübung im Innern eines Staates zu bezeichnen\cite{grundgesetzSouv}.}}

\newglossaryentry{kks}{%
  name={KKS},%
  description={künstliche Währung zum Ausgleich von Preis­niveau-Unterschieden zwischen den Mitgliedstaaten der Europäischen Union; ein Kaufkraftstandard (KKS) entspricht der durchschnittlichen Kaufkraft eines Euro in der Europäischen Union (EU-27)}}

\newacronym{eu}{EU}{Europäische Union}

% generic
\newacronym{api}{API}{Application Programming Interface}
\newacronym{ip}{IP}{Internet Protocol}
\newacronym{ssh}{SSH}{Secure Shell}
\newacronym{cli}{CLI}{Command Line Interface}
\newacronym{htw}{HTW}{Hochschule für Technik und Wirtschaft}
\newacronym{xml}{XML}{Extensible Markup Language}
\newacronym{json}{JSON}{JavaScript Object Notation}
\newacronym{sob}{SoB}{Sentiments of Bundestag}
\newacronym{rest}{REST}{Representational State Transfer}
\newacronym[plural=MDBs, firstplural=Mitglieder des Deutschen Bundestages (MDBs)]{mdb}{MDB}{Mitglied des Deutschen Bundestages}

% group 2
\newacronym{cme}{CME}{Communication Model Extractor}

\makeglossaries

% add literatur to toc
\usepackage[nottoc]{tocbibind}

% graphics and images
\usepackage{graphicx}
\usepackage{subfigure}
\usepackage{wrapfig}

% color packages
\usepackage{color, colortbl}
\definecolor{Gray}{RGB}{216,229,234}
\definecolor{EUBlue}{RGB}{45,172,227}
\definecolor{White}{RGB}{255,255,255}
\usepackage[first=0,last=9]{lcg}
\newcommand{\ra}{\rand0.\arabic{rand}}

% multirow table
\usepackage{multirow}

% footnote package
\usepackage{tablefootnote}

% --------------------------------------------------------------------
% Definitions of title informations
% --------------------------------------------------------------------
\newcommand{\HRule}[1]{\rule{\linewidth}{#1}}

\makeatletter
\def\printtitle{	
    {\centering \@title\par}}
\makeatother			

\makeatletter
\def\printauthor{
    {\centering \large \@author}}
\makeatother

% --------------------------------------------------------------------
% Config Title & Author
% --------------------------------------------------------------------
\title{
\HRule{0.5pt} \\
\LARGE \textbf{\uppercase{Sentiments of Bundestag}}
\HRule{2pt} \\ [0.5cm]
\normalsize \textsc{Graph-basiertes Informationssystem zur Analyse sozialer Interaktion im Deutschen Bundestag}
\\[2.0cm]
\normalsize \today
}

\author{
\normalsize Betreut von
\normalsize Prof. Dr. Thomas Hoppe\\
\normalsize Informationssysteme\\
\normalsize M.Sc. Angewandte Informatik\\
\normalsize Hochschule für Technik und Wirtschaft\\
\normalsize Treskowallee 8, 10318 Berlin, Deutschland\\
}

\begin{document}
% ------------------------------------------------------------------------------
% Maketitle
% ------------------------------------------------------------------------------
\thispagestyle{empty}
\printtitle
  	\vfill
\begin{figure}[H]
    \centering
    \includegraphics[width=200px, keepaspectratio]{logos/bundestag.png}
\end{figure}
  	\vfill
\printauthor		
\newpage

\pagenumbering{roman}
\setcounter{page}{3}

\setcounter{tocdepth}{2}
\tableofcontents
\newpage

\listoffigures
\newpage

\listoftables
\newpage

\pagenumbering{arabic}
\setcounter{page}{6}

\chapter{Einleitung}
\section{Einleitung}\label{sec:08_01_einleitung}
\section{Aufbau der Lösung}\label{sec:01_02_aufbauLoesung}
Das Projekt zur Entwicklung eines graphbasierten Informationssystem für die Analyse sozialer Interaktionen im Deutschen Bundestag, welches \glqq Sentiments Of Bundestag\grqq{} genannt wurde, besteht, wie bereits erwähnt, aus sieben Teilprojekten. Im folgenden Abschnitt wird grob auf die Thematiken der einzelnen Gruppen eingegangen.

\begin{figure}[H]
    \centering
    \includegraphics[width=\textwidth]{images/01-Einleitung/SentimentOfBundestag.png}
    \caption{Aufbau der Lösung}
    \label{fig:aufbauderLösungSOB}
\end{figure}

Die in \autoref{fig:aufbauderLösungSOB} dargestellten Teilprojekte sind hier nun detaillierter aufgelistet:
\begin{itemize}
    \item \textbf{Crawler}: Scannt regelmäßig die Open Data Webseite des Bundestags, sucht, parst und speichern neue Protokollen sowie Stammdaten der Abgeordneten in seiner No-Sql-Datenbank. Ziel ist es hier sicherzustellen, das die DB immer auf dem neusten Stand bleibt
    \item \textbf{Kommunikationsmodell}: Analysiert und erstellt aus den
      Protokollen von Gruppe 1 ein Kommunikationsmodell, welches die möglichen
      Interaktionen im Bundestag abbildet
    \item \textbf{Sentiment-Analyse}: Errechnet die Stimmung der von Gruppe 2 identifizierten Interaktionen und stellt Daten für Gruppe 3 und 4 bereit
    \item \textbf{Interaktion zwischen Abgeordneten}: Aus dem Kommunikationsmodell von Gruppe 2 und die Stimmungsanalyse von Gruppe 3 werden hier Interaktionen zwischen einzelnen Personen (Abgeordneten, Präsident, Gäste, etc.) identifiziert. Erstellt wird wird daraus ein gewichteter Sentiment-Graph zwischen Abgeordneten mit positiven/negativen Gewichtungen
    \item \textbf{Interaktion zwischen Fraktionen}: Aus dem Kommunikationsmodell von Gruppe 2 und den Sentiment-Graph zwischen Personen von Gruppe 4 werden hier Interaktionen zwischen Gruppen von Personen analysiert und in einen Sentiment-Graph zwischen Parteien. Der besteht aus einer Aggregation der Abgeordnetensentiments zu gewichteten Sentiment-Graph der Parteien (Fraktionen, Gruppen, etc.)
    \item \textbf{Graphauswertung}: Die Sentiment-Graphen von den Gruppen 4 und 5 werden hier anhand verschiedener Auswertungsmethoden analysiert und die Ergebnisse davon der nächsten Gruppe (Benutzeroberfläche) zur Verfügung gestellt
    \item \textbf{Benutzeroberfläche}: Ziel ist hier die Realisierung einer interaktiven Benutzeroberfläche zur Darstellung der Ergebnisse
\end{itemize}

Die einzelnen Teilprojekte werden in den nächsten Kapiteln von den jeweiligen Gruppenmitgliedern genauer erläutert.


\chapter{Crawler}
\section{Einleitung}\label{sec:08_01_einleitung}
\section{Grundlagen}\label{sec:02_02_grundlagen}
\section{Anforderungsanalyse und Konzept}\label{sec:08_03_anforderungen_konzept}
\section{Implementierung}\label{sec:02_04_implementierung}
\section{Zusammenfassung und Ausblick}\label{sec:07_05_zusammenfassung}

\chapter{Kommunikationsmodell}
\section{Einleitung}\label{sec:08_01_einleitung}
\section{Grundlagen}\label{sec:02_02_grundlagen}
\section{Anforderungsanalyse und Konzept}\label{sec:08_03_anforderungen_konzept}
\section{Implementierung}\label{sec:02_04_implementierung}
\section{Zusammenfassung und Ausblick}\label{sec:07_05_zusammenfassung}

\chapter{Sentiment Analyse}
\section{Einleitung}\label{sec:08_01_einleitung}
\section{Grundlagen}\label{sec:02_02_grundlagen}
\section{Anforderungsanalyse und Konzept}\label{sec:08_03_anforderungen_konzept}
\section{Implementierung}\label{sec:02_04_implementierung}
\section{Zusammenfassung und Ausblick}\label{sec:07_05_zusammenfassung}

\chapter{Analyse der Interaktion zwischen Abgeordneten}
\section{Einleitung}\label{sec:08_01_einleitung}
\section{Grundlagen}\label{sec:02_02_grundlagen}
\section{Anforderungsanalyse und Konzept}\label{sec:08_03_anforderungen_konzept}
\section{Implementierung}\label{sec:02_04_implementierung}
\section{Zusammenfassung und Ausblick}\label{sec:07_05_zusammenfassung}

\chapter{Analyse der Interaktion zwischen Fraktionen}
\section{Einleitung}\label{sec:08_01_einleitung}
\section{Grundlagen}\label{sec:02_02_grundlagen}
\section{Anforderungsanalyse und Konzept}\label{sec:08_03_anforderungen_konzept}
\section{Implementierung}\label{sec:02_04_implementierung}
\section{Zusammenfassung und Ausblick}\label{sec:07_05_zusammenfassung}

\chapter{Graphauswertung}
\section{Einleitung}\label{sec:08_01_einleitung}
\section{Grundlagen}\label{sec:02_02_grundlagen}
\section{Anforderungsanalyse und Konzept}\label{sec:08_03_anforderungen_konzept}
\section{Implementierung}\label{sec:02_04_implementierung}
\section{Zusammenfassung und Ausblick}\label{sec:07_05_zusammenfassung}

\chapter{Benutzeroberfläche}
\section{Einleitung}\label{sec:08_01_einleitung}
\section{Grundlagen}\label{sec:02_02_grundlagen}
\section{Anforderungsanalyse und Konzept}\label{sec:08_03_anforderungen_konzept}
\section{Implementierung}\label{sec:02_04_implementierung}
\section{Zusammenfassung und Ausblick}\label{sec:07_05_zusammenfassung}

\pagenumbering{roman}
\setcounter{page}{6}

\printbibliography[title={Literaturverzeichnis}]
\addcontentsline{toc}{chapter}{Literaturverzeichnis}
\newpage

\printglossaries
\addcontentsline{toc}{chapter}{Glossar}
\newpage

\end{document}