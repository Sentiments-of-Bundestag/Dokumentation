\section{Einleitung}\label{sec:01_01_einleitung}
\glqq Der raue Ton im Bundestag\grqq{}~\cite{TonImBundesTag2019}. So betitelte die ZDF-Redaktion ihren Beitrag vom 13.11.2019 zur aktuellen Stimmung im Deutschen Bundestag. Dabei ist die Redaktion der Ansicht, dass der Ton seit dem Beitritt der AFD ins Bundestag rauer und aggressiver geworden sei. Sie stützt sich dabei auf zahlreiche Reden, Reaktionen und Kommentare von Abgeordneten verschiedener Parteien. Die bei einer Bundestagssitzung besprochenen Inhalte, sowie Reaktionen und den dadurch erzeugten Stimmungen, werden von Stenographen in Plenarprotokollen bei jeder Sitzung ausführlich dokumentiert. Diese Plenarprotokolle werden frei zugänglich auf der Webeseite des Bundestags bereitgestellt. Des Weiteren sind ebenfalls die Stammdaten öffentlich, welche eine Auflistung aller Abgeordneten darstellen, die jemals im Bundestag vertreten waren.(Open Data~\cite{OpenData2019}) Aus den in den Protokollen verzeichneten Interaktionen lassen sich Vernetzungen zwischen einzelnen Personen und Gruppen von Personen erkennen (Fraktionen, Parteien, etc.), welche dazu dienen einen Kommunikationsgraphen (Kommunikationsmodell) aufzustellen. Aufgrund dieses Zusammenhangs, im Rahmen des Moduls Informationssysteme vom Master-Studiengang Angewandte Informatik (AI) der HTW Berlin und auf Vorschlag vom Prof. Dr. Thomas Hoppe wurde dieses Projekt gestartet. Aufgabe des Projektes ist die Untersuchung der Aussage der ZDF-Redaktion mithilfe eines graphbasierten Informationssystems, welches die sozialen Interaktionen im Deutschen Bundestag analysiert. Aus dem von den Plenarprotokollen abgeleiteten Kommunikationsmodell wird ein Sentiment-Graph entwickelt, welcher dazu dient die Beziehungen zwischen einzelnen Abgeordneten, sowie verschiedener Parteien zu analysieren, welche im Anschluss graphisch dargestellt werden. Zur Lösung dieser Aufgabe wurde das Projekt in sieben Teilprojekte unterteilt. Jede Gruppe, welche im Durchschnitt aus drei Studierenden bestand, hat sich mit der jeweiligen aufgetragenen Thematik beschäftigt und diese bearbeitet. Auf die Thematik der jeweiligen Teilprojekte sowie auf deren konkreten Zielsetzung als auch Umsetzung wird im nächsten Abschnitt genauer eingegangen.
