\section{Einleitung}\label{sec:01_01_einleitung}
\citetitle{TonImBundesTag2019}~\cite{TonImBundesTag2019}. So betitelte die
\enquote{\gls{zdf}}-Redaktion ihren Beitrag vom 13.11.2019 zur aktuellen Stimmung im
Deutschen Bundestag. Dabei ist die Redaktion der Ansicht, dass der Ton
seit dem Beitritt der \gls{afd} in den Bundestag rauer und aggressiver geworden
sei. Sie stützt sich dabei auf zahlreiche Reden, Reaktionen und Kommentare
von sowohl Abgeordneten dieser Partei als auch anderer Parteien,
dessen Inhalte, vermittelten und erzeugten Stimmungen, sowie Reaktionen
darauf, welche von Stenographen in Plenarprotokollen bei jeder Sitzung
aufgezeichnet und dokumentiert werden. Diese Plenarprotokolle werden als \enquote{\gls{xml}}-
oder ZIP-Dateien gemeinsam mit allen Stammdaten auf der Webseite des Bundestags
(\citetitle{OpenData2019}~\cite{OpenData2019}) veröffentlicht. Die in den Protokollen
verzeichneten Interaktionen bilden Netze \todo{Vielleicht weiter ausholen, um
das verständlich zu erklären? Insbesondere bilden Interaktionen von sich aus keine Netze}
zwischen einzelnen Personen und Gruppen von Personen (Fraktionen, Parteien,
etc.), die als Kommunikationsgraphen (Kommunikationsmodell) betrachtet werden
können. In diesem Zusammenhang, im Rahmen des Moduls Informationssysteme des
Master-Studiengangs \gls{ai} an der \gls{htw} Berlin und auf Vorschlag vom
Prof. Dr. Thomas Hoppe haben wir uns zur Untersuchung der Aussage der
\gls{zdf}-Redaktion vorgenommen, ein Graph
basiertes Informationssystem zur Analyse sozialer Interaktion im Deutschen Bundestag
zu entwickeln. Ausgehend von den Plenarprotokollen und dem daraus
abgeleiteten Kommunikationsmodell baut das System einen Sentiments-Graph auf,
der die Interaktionen innerhalb des Bundestags abbildet, die Stimmung zwischen
Abgeordneten und verschiedenen Gruppen analysiert (bzw. misst) und grafisch
darstellt. Zur Lösung dieser Aufgabe wurde das Projekt in sieben Teilprojekte
unterteilt, mit denen sich sieben Gruppen von Studierenden auseinandergesetzt
haben. Auf die einzelnen Teilprojekte sowie auf deren konkrete Zielsetzungen
wird im nächsten Abschnitt genauer eingegangen.