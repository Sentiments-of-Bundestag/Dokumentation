\section{Einleitung}\label{sec:01_01_einleitung}
Einen Beitrag vom 13.11.2019 zur aktuellen Stimmung im Deutschen Bundestag betitelt die ZDF-Redaktion mit \glqq Der raue Ton im Bundestag\grqq{}~\cite{TonImBundesTag2019}. Sie ist der Ansicht, dass der Umgangston seit der Wahl der AfD in den Bundestag rauer und aggressiver geworden ist. Die Redaktion stützt sich dabei auf zahlreiche Reden, Reaktionen und Kommentare von Abgeordneten verschiedener Parteien.
Die bei einer Bundestagssitzung besprochenen Inhalte und Reaktionen werden von Stenographen in Plenarprotokollen ausführlich dokumentiert. Diese Protokolle sind frei zugänglich auf der Webseite des Bundestages. Auch die Stammdaten, eine Auflistung aller Abgeordneter, die jemals im Bundestag vertreten waren (Open Data~\cite{OpenData2019}), sind öffentlich.
Im Rahmen des Moduls Information Systems im Master-Studiengang Angewandte Informatik der HTW Berlin und auf Vorschlag von Prof. Dr. rer. nat. Thomas Hoppe wurde ein Projekt gestartet, welches auf Basis der Plenarprotokolle die Aussage der ZDF-Redaktion prüfen möchte. Dies soll mithilfe eines graphbasierten Informationssystems, welches die sozialen Interaktionen im Deutschen Bundestag umfasst, gelingen. Aus den Plenarprotokollen sollen durch ein Kommunikationsmodell die Interaktionen ermittelt und anschließend die jeweilige Stimmung errechnet werden. Auf Basis dieser Daten sollen weitere Berechnungen für einzelne Abgeordnete oder ganze Frktionen angestellt und als ein Sentiment-Graph persistiert werden. Dieser Graph soll als Grundlage für eine ausführliche Visualisierung dienen. 
Zur Lösung dieser Aufgaben wurde das Projekt in sieben Teilprojekte unterteilt. Jede Gruppe, welche im Durchschnitt aus drei Studierenden bestand, hat sich mit der jeweiligen aufgetragenen Thematik beschäftigt und diese bearbeitet. Auf die Thematik der jeweiligen Teilprojekte sowie auf deren konkreten Zielsetzung und Umsetzung wird im nächsten Abschnitt genauer eingegangen.
