\section{Einleitung}\label{sec:02_01_einleitung}
Die Umsetzung eines graphbasierten Informationssystems zur Analyse sozialer
Interaktionen im Deutschen Bundestag setzt voraus, dass aktuelle Informationen und Berichte über die Debatten im Bundestag zu jeder Zeit zur Verfügung stehen. Diese Informationen sowie Stammdaten von Abgeordneten und anderen Sitzungsteilnehmern werden regelmäßig auf der Seite des Bundestags veröffentlicht und sind für alle interessierten Anwender frei zugänglich~\cite{OpenData2019}. Das Durchsuchen der Webseiten, das Vergleichen, Herunterladen und Parsen der Protokolle, als auch der Stammdaten, lässt sich mithilfe eines Computerprogramms automatisieren. Ein solches Programm wird als Crawler bezeichnet. Im Allgemeinen besteht die Aufgabe eines Crawlers darin sich wiederholende Operationen, wie zum Beispiel der Download oder die Indexierung einer Webseite, soweit wie möglich zu systematisieren und ohne manuellen Eingriff regelmäßig auszuführen. Ein Crawler funktioniert in der Regel wie ein Bot und kann zur Extraktion bestimmter Informationen von einer oder mehreren Webseiten verwendet werden. Des Weiteren kann dieser auch eine komplette ggf. leicht abgewandelte Replikation des Inhalts einer Webseite erstellen. Im zweiten Fall geht es um einen sogenannten Scraper. Zur Sammlung der vom graphbasierten Informationssystems benötigten Daten wird ein Crawler eingesetzt, der wie bei einem Scraper eine bestimmte Seite in einer festgelegten Frequenz herunterlädt und aus dieser Links für spezifische Datentypen extrahiert. Abhängig davon, ob die gesammelten Links bereits bekannt sind oder nicht, werden die entsprechenden Dateien gedownloadet, geparst und anschließend in einer Datenbank gespeichert. Bevor der Crawler implementiert wird, ist es zunächst wichtig genaue technische Anforderungen aufzustellen, sowie eventuelle Schwierigkeiten, in diesem Fall auch speziell für die Bundestag-Seite, zu identifizieren. Daraus wird ein Konzept zur Lösung dieser Aufgabe erarbeitet. Auf die beiden zuvor genannten Punkte, die Implementierung des vorgeschlagenen Konzepts und die Bereitstellung der vollständigen Lösung, wird in den folgenden Abschnitten dieses Dokuments näher eingegangen.