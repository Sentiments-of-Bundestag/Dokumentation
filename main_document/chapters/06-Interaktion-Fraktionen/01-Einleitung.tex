\section{Einleitung}\label{sec:06_01_einleitung}
Für die Analyse der Interaktionen zwischen Fraktionen sollen sowohl Interaktionen ausgewertet werden, die explizit von einer Fraktion an eine andere Fraktion gerichtet sind, als auch Interaktionen zwischen Fraktionsmitgliedern auf Fraktionsebene betrachtet werden. Ziel der Analyse soll sein, alle protokollierten Interaktionen auszuwerten und in Form eines Graphen, der die Sentiments zwischen Fraktionen darstellt abzulegen.
Fraktionen spielen eine entscheidende Rolle im Deutschen Bundestag. Sie dürfen u.a. Gesetzentwürfe bzw. Änderungsanträge von solchen einbringen, kleine und große Anfragen im Bundestag stellen, und Sondersitzungen des Bundestags erzwingen. Das bedeutet, dass Fraktionen und deren gegenseitigen Aussprachen ``die politische Willensbildung maßgeblich mitbestimmen'' \cite{factionsbpp}.
Dabei ist zu beachten, dass Fraktionsmitglieder ein freies Mandat bekleiden und damit zu abweichenden Meinungen gegenüber ihrer Fraktion befugt sind. \cite{abgeordneterbpp} Daraus ergibt sich für die Aufgabenstellung das Problem, inwiefern Interaktionen zwischen Fraktionsmitgliedern in einen Graphen, der die Sentiments zwischen Fraktionen zeigen soll einfließen können. Spiegelt die Meinung eines Fraktionsmitglieds genau die Meinung der zugehörigen Fraktion wider? Können Interaktionen, die im Protokoll als von einer Fraktion ausgehend markiert sind auf alle ihre Mitglieder übertragen werden?

Wie mit dieser Frage umgegangen wurde wird im weiteren Verlauf erläutert. Weiterhin wird beschrieben, wie die gefundene Lösung technisch umgesetzt wurde, indem auf Anforderungen, genutzte Technologien und Endergebnisse eingegangen wird. Abschließend werden Optimierungsmöglichkeiten der Applikation dargestellt.