\section{Einleitung}\label{sec:03_01_einleitung}
Im Rahmen des Projekts \enquote{Sentiments of Bundestag} (SoB) \todo{glos} sollten
Bundestagsprotokolle analysiert werden, um den \enquote{Ton} des Bundestags zu
Bewerten. Dieses Teilprojekt von Gruppe 2 ist für das \enquote{Kommunikationsmodell}
(\enquote{Communication Model Extractor}, CME \todo{glos}) zuständig. Es ist
der letzte Service der sich mit den Rohdaten der Bundesregierung beschäftigt.
Aufbauend auf den hier erstellten Daten werden anschließend von Gruppe 3 die
Sentimente berechnet.

\subsection{Zielstellung}

Dieser Service soll aus den Open Data XML-Dateien der 19. Wahlperiode des
Bundestages bzw. der leicht abgewandelten Variante dieser Daten, welche von
Gruppe 1 in Form von JSON-Dateien zur Verfügung gestellt werden,
Interaktionen zwischen Fraktionen und oder Abgeordneten extrahieren.

Für diese Extraktion werden die Rohdaten analysiert und darauf aufbauend ein
passendes Kommunikationsmodell gewählt. Ebenfalls wird ein Datenmodell
entwickelt, welches zur Speicherung und Weitergabe an die folgenden Gruppen
dient.

\subsection{Anforderungsdefinition}

Auf Grundlage der im vorherigen Abschnitt beschriebenen Zielsetzung sowie der
Diskussionen in den Plenarsitzungen während des Semesters, erfolgt in diesem
Kapitel eine Zusammenfassung von Anforderungen, welche an das zu entwickelnde
System gestellt werden. Diese werden in funktionale und nicht funktionale
Anforderungen sowie in Muss-, Soll- und Kann-Kriterien unterteilt.

Funktionale Anforderungen sind dabei Anforderungen, die beschreiben, was die
Software können soll. Nicht-funktionale Anforderungen dagegen beschreiben wie
die Software funktionieren soll.

\begin{table}[H]
    \renewcommand{\arraystretch}{2.5} % Default value: 1
    \centering
    \begin{tabularx}{\textwidth}{c|c|X}
                                & \textbf{FR01} & \noindent\parbox[c]{\hsize}{
                                                  Ein passendes Kommunikationsmodell und
                                                  Datenmodell muss gewählt werden} \\
                                & \textbf{FR02} & \noindent\parbox[c]{\hsize}{
                                                  Das Ausgabeformat von Gruppe 1 muss eingelesen
                                                  werden können} \\
                                & \textbf{FR03} & \noindent\parbox[c]{\hsize}{
                                                  Interaktionen basierend auf den Kommentaren
                                                  zu Redebeiträgen müssen extrahiert werden} \\
                                & \textbf{FR04} & \noindent\parbox[c]{\hsize}{
                                                  Extrahierte Interaktionen müssen für den
                                                  Zugriff späterer Gruppen persistiert werden} \\
        \textbf{Muss-Kriterien} & \textbf{FR05} & \noindent\parbox[c]{\hsize}{
                                                  Spätere Gruppen müssen auf die persistierten
                                                  Nachrichten zugreifen können} \\
                                & \textbf{FR06} & \noindent\parbox[c]{\hsize}{
                                                  Gruppe 1 muss die Möglichkeit haben, uns über
                                                  die Verfügbarkeit neuer Daten zu
                                                  benachrichtigen} \\
                                & \textbf{FR07} & \noindent\parbox[c]{\hsize}{
                                                  Gruppe 3 muss von uns benachrichtigt werden,
                                                  wenn neue Daten zur Verfügung stehen} \\
                                & \textbf{FR08} & \noindent\parbox[c]{\hsize}{
                                                  Parteien und Abgeordnete müssen über Sitzungen
                                                  hinweg eindeutig zuordenbar sein} \\
                                & \textbf{FR09} & \noindent\parbox[c]{\hsize}{
                                                  Daten von Gruppe 1 müssen aus deren MongoDB
                                                  ausgelesen werden können} \\
        \hline

        \textbf{Soll-Kriterien} & \textbf{FR10} & \noindent\parbox[c]{\hsize}{
                                                  Das Open Data Ausgabeformat des Bundestags soll
                                                  eingelesen werden können} \\
        \hline

        \textbf{Kann-Kriterien} & \textbf{FR11} & \noindent\parbox[c]{\hsize}{
                                                  Interaktionen innerhalb der Redebeiträge können
                                                  extrahiert werden} \\

    \end{tabularx}
    \label{tab:03_requirements}
\end{table}

\subsection{Wahl des Kommunikationsmodell}
FR01 erfordert die Auswahl eines passenden Kommunikations- und Datenmodells.
Aus diesem Grund wurden zu Beginn des Semesters bereits vorliegenden
Plenarprotokolle im Open Data Format der Bundesregierung untersucht.

Bei dieser Untersuchung zeigte sich, dass die für uns interessanten Daten so
angeordnet sind, dass jede Sitzung in Tagesordnungspunkte unterteilt ist,
welche wiederum aus einer Vielzahl von Reden besteht. Eine solche Rede
enthält dabei Informationen über den Redner, die in einzelne Absätze
zerteilte Rede und optionale Kommentare anderer Parlamentarier oder
Fraktionen zu diesen Absätzen. So sind in diesen Strukturen Interaktionen
erkennbar, z.~B. in den Kommentaren an den Redner gerichtete Interaktionen oder
auch in den Absätzen Interaktionen zwischen dem Redner und anderen
Parlamentariern.

Basierend auf diesen Erkenntnissen und in Absprache mit Gruppe 3, 4 und 5,
welche für die Sentimentanalyse sowie die Auswertungen zwischen Fraktionen und
Abgeordneten verantwortlich sind, wurde ein Sender-Empfänger-Modell gewählt.
In diesem kann der Sender und Empfänger jeweils entweder ein
Bundestagsabgeordneter oder eine Fraktion sein. Auf Basis des Inhalts der
Nachricht, die vom Sender zum Empfänger geschickt wird, wird später von
Gruppe 3 das Sentiment ermittelt.

\section{Umsetzung}\label{sec:03_02_umsetzung}
\todo{text? or drop first subsection?}

\subsection{Communication Model Extractor (CME)}
Nachdem die Anforderungen definiert wurden, wurde sich im Team die Struktur
überlegt, wie die Applikation aufgebaut werden soll, um effizient die Aufgabe
zu erfüllen.

Es wurde sich für die Programmiersprache Python entschieden, da sich alle
Teammitglieder damit gut auskennen. Da das Kommunikationsmodell außerdem an
zweiter Stelle der Pipeline steht, schien es sinnvoll, schnelle Ergebnisse
vorweisen zu können, was ebenfalls für Python spricht.

Aus den funktionalen Anforderungen FR5, FR6 und FR7 ergibt sich, dass eine
Kommunikation zu den Gruppen 1 und 3 nötig ist. Aufgrund hoher Kompatibilität
wurde sich für eine REST-Schnittstelle entschieden. Diese API wurde in dem
Python-Modul \enquote{api.py} mithilfe der FastAPI-Bibliothek implementiert. Sie
beantwortet Anfragen zu Protokollen (Sessions) Fraktionen (Factions) und
Bundestagsabgeordneten (MDBs). Außerdem wird eine SWAGGER-Dokumentation
bereitgestellt.

Zur Steuerung der Anwendung wird ein CLI, welches im \enquote{cli.py}-Modul umgesetzt
wurde, in Form der cme-Applikation zur Verfügung gestellt. Dieses ist in
verschiedene Unterkommandos aufgeteilt. Mithilfe dieser ist es z.B. möglich
Protokolle manuell aus dem lokalen Speicher einzulesen (XML- oder JSON-Format)
oder den Server zu starten. Mit \enquote{init} werden erstmalig Bundestagsabgeordnete
eingelesen. \enquote{dump} ermöglicht die Interaktion mit der Datenbank, um einzelne
Dokumente daraus zu extrahieren.

Um die Anwendungslogik von der Business-Logik zu trennen, wird der \enquote{Controller}
eingeführt. Es ist das zentrale Modul, welches die Prozesse koordiniert.

Für die generische Verarbeitung der Rohdaten in den verschiedenen Datenformaten
von Gruppe 1 (FR02) und der Open Data Protokolle des Bundestags (FR10) wurde
das \enquote{Data}-Modul implementiert. Dieses wandelt die Rohdaten unter anderem in
sogenannte InteractionCandidates um, welche dann mithilfe des \enquote{Extraction}-
Moduls verarbeitet und ausgewertet werden (FR03 und FR11). So extrahiert dieses
aus den InteractionCandidates die stattgefundenen Interaktionen basierend auf
dem Sender-Empfänger-Modell und legt diese in einem sogenannten Transcript ab.
Dieses Transcript wird anschließend mithilfe des \enquote{Database}-Moduls, welches für
jeglichen Datenbankzugriff zuständig ist, in der Datenbank abgelegt.

Zuletzt enthält \enquote{Domain} Datenmodelle für verschiedene Objekte, wie z.~B. MDB,
Faction \& Interaction. Es wird von so gut wie jedem Modul benutzt.

Die soeben beschriebenen Module und der damit verbundene Kontrollfluss zwischen
diesen ist in \autoref{fig:03_project_structure} dargestellt.

\begin{figure}
    \begin{center}
        \includegraphics[width=\textwidth]{example-image-a}
    \end{center}
    \caption{TODO: write me}
    \label{fig:03_project_structure}
\end{figure}
\todo{figure caption}

\subsection{Eingabeformate}
Wie in den Anforderungen FR02 und FR10 definiert, muss als Eingabeformat
einerseits das JSON Format der Gruppe 1 und andererseits soll auch das
Open Data XML Format des Bundestags unterstützt werden. So wurden
spezielle Parser für die jeweiligen Formate entwickelt, deren Ausgabe gleich
formatiert ist und somit für die nächste Instanz eine einheitliche
Schnittstelle bildet. Dieses Ausgabeformat enthält dabei neben Metadaten zur
Sitzung selbst, wie z.~B. den Sitzungsstartzeitpunkt oder die Sitzungsnummer,
sogenannte InteractionCandidates. Diese gruppieren, wie der Name bereits
nahelegt, Daten, welche eventuelle für die Auswertung relevante Interaktion
enthalten. So gruppiert ein InteractionCandidate den Redner (Speaker),
einen Absatz von dessen Rede (Paragraph) und den eventuell darauf folgenden
Kommentarblock (Comment).

\subsection{Methodik der Kommunikations-Analyse}
Die zu analysierenden Daten (InteractionCandidates) bieten zwei verschiedene
Kommunikationstypen als Datengrundlage für die Erkennung von Interaktion:

\begin{enumerate}
    \item \textbf{Kommentare (FR03)}: werden von den Redeteilen gesondert
        dargestellt und mit zusätzlicher Information bezüglich der
        kommunikativen Einordnung des Beitrags versehen.
    \item \textbf{Redeteile (FR10)}: werden in den Protokollen als
        Paragraphen oder sinngemäß als Absätze aufgeführt.
\end{enumerate}

Aufgrund der Unterschiedlichkeit mussten beide Typen gesondert analysiert werden.

\subsubsection{Kommentare}

Kommentare werden nach rigiden strukturellen Regeln in den Protokollen der
Bundestagssitzungen erfasst und liefern Informationen zum Sender des
Kommentars. Dies kann eine oder mehrere Fraktionen oder eine oder mehrere
an der Sitzung teilnehmende Personen sein. Dabei lassen sich Kommentare in
drei Kategorien unterteilen:

\begin{itemize}
    \item \textbf{Publikumsresonanzen}: Z.~B. Beifall oder Entrüstung des
        Senders, bei der eine genaue Zitierung durch die Stenographen nicht
        möglich ist. Hier folgen Sender durch Präpositionen abgetrennt auf
        den Kommentarinhalt. Bsp.: \enquote{Beifall bei der CDU};
        \enquote{Zurufe von der Linken und Grünen}

    \item \textbf{Zitate}: Hier wird der Wortlaut eines Kommentars
        wiedergegeben. Bei diesen Kommentaren wird die kommentierende Person
        dem Wortlaut vorangestellt und durch einen Doppelpunkt abgetrennt.
        Bsp.: \enquote{Dr. Klaus Hermann: Ganz toll gemacht}; \enquote{Dr.
        Alexander Gauland: Schwachsinn}

    \item \textbf{Beobachtungen der Protokollanten}: Spezielle Situationen in
        denen ein oder mehrere Sitzungsteilnehmer nennenswerte Handlungen
        vornehmen. Auch hier wird in der Regel der Sender der Nachricht
        vorangestellt, wenngleich keine Trennung durch einen Doppelpunkt
        erfolgt. Bsp.: \enquote{Manfred von Kuchenhausen verlässt die Sitzung}

\end{itemize}

Je nach erkannter Kommentarform kommen also verschiedene strukturelle
Analyse-Vorgänge für die Feststellung von Sendern zum Einsatz.

\paragraph{Kommentare mit abweichender Struktur}

Es gibt weitere Eigenschaften, die die Extraktion der Interaktionen
erschweren. So können pro Kommentar-Abschnitt mehrere Strukturen mit
unterschiedlichen Inhalten auftreten, oder es werden insbesondere bei
Publikumsresonanzen mehrere Absender aufgeführt (etwa
\enquote{Beifall von SPD und die Linke}), wodurch zusätzlich eine korrekte
Auftrennung der Sender auf Basis von Konjunktionen bzw. Interpunktion
notwendig wird. Außerdem gibt es gelegentlich sekundäre Kommentare, bei denen
auf vorherige Kommentare reagiert wird. Die einzigen zuverlässigen
Formatierungen, die auf solche Verhältnisse hinweisen, sind die expliziten
grammatikalische Struktur nach der Namensnennung:
\enquote{an [neuer Empfänger] gerichtet} bzw. \enquote{zur [Ziel-Fraktion] gewandt}
wirken innerhalb von Kommentaren als klares Merkmal für einen vom Redner
abweichenden Empfänger. Das Schlüsselwort \enquote{Gegenruf} wird darüber
hinaus häufig verwendet, um auf einen Dialog zwischen Kommentierenden
hinzuweisen - da in diesen Fällen allerdings der Kommentarinhalt oft
mindestens teilweise weiterhin auf den Redner abzielt, wurde bei solchen
Kommentaren ebenfalls auf den Redner als Empfänger referenziert. Diese
Problematik könnte in Fortsetzungen der Arbeit eventuell präziser behandelt
werden, da in längeren Kommentar-Dialogstrukturen recht häufig auf die
eindeutige Empfängerbezeichnung verzichtet wird nachdem diese initial
eingeleitet werden, wodurch in unserer Methode viele für menschliche Leser
offensichtliche Interaktionen nicht als solche registriert wurden. Nicht
zuletzt werden die genannten syntaktischen Strukturen gelegentlich durch
Rechtschreib- oder Zeichensetzungsfehler durchbrochen, wobei in diesen Fällen
die gesamte Interaktion verloren geht.

\subsubsection{Redeteile}
Redeteile besitzen weniger Struktur als Kommentare, da sie nicht den Regeln der
Stenographen unterliegen, sondern des individuellen Redners. So muss der
Empfänger einer Aussage erst ermittelt werden. Redeteile wurden daher auch nur
als Interaktionen gewertet, sobald Fraktionen oder Mitglieder des Bundestags
als mögliche Interaktionsempfänger im Fließtext erkannt wurden.

Die Erkennung von MDBs erfolgt über die Feststellung bekannter, eindeutiger
Nachnamen, denen eine formale Anrede (Herr/Hr., Frau/Fr., Kollege, Doktor)
vorangeht. Parteien wurden anhand von ihren Namen bzw. deren Abkürzungen,
beziehungsweise durch gängige aber eindeutige Kurzformen identifiziert. Die
festgestellte Person oder Partei wurde anschließend als Empfänger der Nachricht
behandelt, während die sprechende Person, also der durch den
Sitzungspräsidenten zuletzt eingeleitete Redner, die Rolle des Absenders
einnahm.

Die Empfängerermittlung bei Redeanteilen verblieb in der entstandenen
Implementierung in einem recht rudimentären Zustand, da viele der inhärenten
Probleme des Ansatzes nicht behandelt werden konnten. So konnte etwa die
Unterscheidung zwischen mehreren MDBs mit gleichen Nachnamen oder die
kolloquiale Verwendungen von Farben (\enquote{Rot} für SPD oder die Linke) oder
Begriffen (\enquote{Liberale} stellvertretend für die FDP) als Substitut für Fraktionen
nicht behandelt werden. Um diese Problematiken zu umgehen wurden in der
Paragraphen-Analyse nur solche Interaktionen gewertet, welche sich an eindeutig
identifizierbare MDBs oder Fraktionen richten. Nachnamen wie \enquote{Müller}, die
mehrere MDBs bezeichnen können, aber auch in anderen Sachverhalte eingesetzte
Begriffe, wie \enquote{Union}, wurden somit aus der Schlüsselwort-Liste verbannt, mit
der die Redeabschnitte abgetastet wurden. Auch die Nennung von Nachnamen als
Referenz, etwa bei Beiträgen, die sich auf Morddrohungen gerichtet an im
Bundestag auftretende Personen thematisierte, wurde im späteren Verlauf der
Arbeit als Grund für Mängel in der Ergebnissen \todo{fehler?} identifiziert, in diesem Fall in
der Form von fehlerhaften Interaktionen. Durch diese Probleme wurde eine
unbestimmt große Menge von Fehlern durch die Redeteil-Analyse produziert,
sowohl in nicht erkannte Interaktionen als auch fälschlicherweise als
Interaktionen kategorisierte Beiträgen, was mit Sicherheit eine Verzerrung der
Projektergebnisse, aber gleichzeitig auch ein wesentliches
Verbesserungspotential für die mögliche Fortsetzungen dieses Teilprojekts
darstellt. Gleichzeitig sollte nicht außer Acht gelassen werden, dass
Interaktionen in Redeteilen nach unseren Messungen einen recht geringen Anteil
der Interaktionen im Bundestag ausmachen. Die Steigerung der Anzahl erkannter
Interaktionen durch unsere Methode der Redebeitrag-Analyse betrug für die
bisherigen Protokolle der 19. Wahlperiode lediglich 8\% gegenüber der alleinigen
Betrachtung der Kommentare.

\subsection{Ermittlung von teilnehmenden Personen}
Personen müssen über Sitzungen hinweg eindeutig identifizierbar sein, damit
nachfolgende Gruppen Informationen zu Parteizugehörigkeit und Namen korrekt
zuweisen können (FR08). Personen werden in der \enquote{mdb} Collection der MongoDB
gesammelt. Initialisiert werden kann diese Liste mithilfe der Stammdaten der
Bundestagsabgeordneten des Bundestags. Diese als XML zur Verfügung gestellte
Liste enthält alle relevanten Informationen, die bei der Identifizierung
helfen (Namen und Namensänderungen, Parteizugehörigkeit mit von-bis
Datumsangabe, Titel etc.).

Da die Stammdaten unvollständig sind, auch weil Gastredner Reden halten oder
Bundestagsabgeordnete in den Stammdaten fehlen, müssen weitere Redner während
der Analyse hinzugefügt werden. Gerade in den Kommentaren können meist Vor-
und Nachnamen extrahiert werden, um ein Abgleich mit den MDB-Daten
vorzunehmen. Leider werden Namen oft inkonsistent geschrieben (z.~B. wird
der zweite Vorname manchmal ausgeschrieben, abgekürzt oder weggelassen) oder
die Stenographen vertippen sich. Auch wird das Format der Kommentarsektion
nicht immer eingehalten.

Weil wir fehlende Redner und Rechtschreibfehler nicht unterscheiden können,
gelangen doppelte Einträge in die Datenbank. Wir haben Mechanismen eingebaut,
um solche Fehler zu erkennen, können Sie aber nicht ganz ausschließen.

\subsection{Infrastruktur Setup mit Docker}
Um ein einfaches Deployment sicherzustellen, wurde sowohl die mongodb als
auch der Service selbst mit Docker bzw. docker-compose abstrahiert.

Die Standardkonfiguration des mongodb Containers wurde lediglich um
Berechtigungen erweitert.

\subsubsection{Datensicherheitsvorfall}

Während der Projektimplementierung wurde der MongoDB Server, auf dem die
Kommunikationsdaten abgelegt wurden, aufgrund von Fehlkonfigurationen
kompromittiert.

Die Firewall wurde zwar korrekt konfiguriert, der Docker-Daemon öffnet mit
seinen Root-Rechten Container jedoch standardmäßig nach außen. Außerdem wurde
kein Benutzername oder Passwort für die Datenbank gesetzt. Somit konnte ein
Angreifer den offenen MongoDB-Port finden, die Datenbank löschen und eine
Erpressernachricht hinterlassen - ein klassischer Ransomware-Angriff.

Weil keine der Daten prägnant waren (da sie ohnehin im Netz frei verfügbar
waren), konnte über den Verlust hinweg gesehen werden.

Der zuständige Laboringenieur wurde kontaktiert und der Server zurückgesetzt.
Nach einer Analyse der Fehler wurden Zugangsdaten für die Datenbank angelegt
sowie Docker umkonfiguriert.

Danach wurden die Maßnahmen getestet, um sicherzustellen, dass nur noch
Nutzer mit SSH und Datenbank-Zugangsdaten Zugriff haben.

Zukünftig wäre es sinnvoll den Docker-Daemon als Nicht-Root-User laufen zu
lassen, um die Anwendung weiter abzusichern.

\subsection{Schnittstelle für Zugriff auf den Communication Model Extractor}

Um den gewünschten Workflow einer Pipeline zu ermöglichen, entschieden wir
uns eine API anzubieten, wodurch die Kommunikation zur vorherigen und
folgenden Gruppe realisiert wurde.

\begin{figure}[ht]
    \begin{center}
        \includegraphics[width=\textwidth]{example-image-a}
    \end{center}
    \caption{TODO: write me}
    \label{fig:03_api_call_flow}
\end{figure}
\todo{figure caption}

Es wurde sich mit Gruppe 1 darauf geeinigt, dass der CME einen Endpunkt
anbietet, um die Information zu erhalten, wenn neue Protokolle aus dem
Bundestag in deren Datenbank kreiert wurden. Zusätzlich bekamen wir
Zugangsdaten, um auf deren Datenbank zugreifen zu können. So werden uns die
Protokoll-ID's an \url{http://infosys2.f4.htw-berlin.de:9001/cme/data/} gesendet
und der CME holt sich die gewünschten Protokolle zu einem späteren Zeitpunkt
aus deren Datenbank. Danach kann die Auswertung geschehen.

Um der darauffolgenden Gruppe Bescheid zu geben, wird nach dem Prozess der
Auswertung ebenfalls eine HTTP Anfrage gesendet, in der über die neuen
Protokoll-ID's informiert wird. Unsere API bietet dafür verschiedene Endpunkte
an, um an alle notwendigen Informationen zu gelangen.

Diese Endpunkte sind alle in einem Endpunkt festgehalten, die die API
dokumentiert und hier abgerufen werden kann:
\url{http://infosys2.f4.htw-berlin.de:9001/cme/doc/docs}

Demnach enthält der CME folgende Endpunkte:
\begin{itemize}
    \item /cme/data/sessions/ - damit wird eine Liste aller existierenden
        Sitzungen zurückgegeben
    \item /cme/data/session/\{session\_id\} - um eine bestimmte Sitzung mit den
        ermittelten Interaktionen zu erlangen
    \item /cme/data/mdb - nützlich, um nach Mitgliedern des Bundestags zu
        suchen, mithilfe der Query-Parameter mdb\_id, speaker\_id, forename
        \& surname kann gefiltert werden
    \item /cme/data/faction - damit können alle existierenden Fraktionen mit
        jeweiliger ID abgerufen werden
\end{itemize}

Der Server ist nur über das HTW-Netzwerk erreichbar. Zusätzlich ist die API mit
Basic Authentication geschützt. Dadurch kann nur auf die Daten zugegriffen
werden, wenn sich mit Username und Passwort authentifiziert wird.

Es wurden drei Clients, sprich User, angelegt, die auf die API zugreifen
dürfen. Je einer für die anderen Gruppen crawler\_client \& sentiment\_client und
cme\_admin für unser Team, um Testen zu können und bei Bugs der Ursache auf den
Grund gehen zu können.

\section{Fazit}\label{sec:03_05_fazit}

Abschließend konnten die meisten Anforderungen erfolgreich umgesetzt werden.

Der Service bietet Schnittstellen zur vorherigen und nachfolgenden Gruppe und
sowie automatisierte Verarbeitungs-Mechanismen (notify). Auch die frühzeitige
Bereitstellung einer akzeptierten Kommunikations-Modellierung auf Basis der XML-
Rohdaten konnte realisiert werden. Der Großteil des Datenbestands wird
analysiert, sodass sowohl Kommentare als auch Redeteile auf Kommunikationen
überprüft werden.

Allerdings verwirft der Service, aufgrund von Inkonsistenzen in der Syntax und
Rechtschreibung sowie aufgrund von fehlenden kontextbasierten Analyse-
Mechanismen, potentielle Interaktionen. Während in Kommentaren primär durch
Abweichungen von Protokoll-Syntax und Rechtschreibfehler Probleme entstanden,
so fehlte in den Redeteilen oft die nötige Information um die Empfänger von
Interaktionen eindeutig zu bestimmen. Daher mussten Interaktionen verworfen
werden und teils wurden in Kommentaren inkorrekt genannte Personen als
Personenduplikate in den Datenbestand aufgenommen.

Daher wurde das Ziel der Gruppe \enquote{Kommunikationsmodell} in Funktionalität und
Integration zwar erreicht, allerdings wird deutliches Verbesserungspotential
für die Extraktion von noch mehr Interaktionen und die Eliminierung von
Artefakten gesehen

Die ausgelieferte Deployment-Strategie und die Modulbeschreibungen in dieser
Dokumentation sollten bei der Weiterentwicklung des Projekts hilfreich sein.