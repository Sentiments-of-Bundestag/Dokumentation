\section{Einleitung}\label{sec:03_01_einleitung}
\subsection{Zielstellung}

\subsection{Anforderungsdefinition}

\begin{table}[H]
    \renewcommand{\arraystretch}{2.5} % Default value: 1
    %\renewcommand\baselinestretch{0.5}
    \centering
    \begin{tabularx}{\textwidth}{c|c|X}
    %\begin{tabular}{l|l|p{0.67\textwidth}}
                                & \textbf{FR01} & \noindent\parbox[c]{\hsize}{
                                                  Ein passendes Kommunikationsmodell und
                                                  Datenmodell muss gewählt werden} \\
                                & \textbf{FR02} & \noindent\parbox[c]{\hsize}{
                                                  Das Ausgabeformat von Gruppe 1 muss eingelesen
                                                  werden können} \\
                                & \textbf{FR03} & \noindent\parbox[c]{\hsize}{
                                                  Interaktionen basierend auf den Kommentaren
                                                  zu Redebeiträgen müssen extrahiert werden} \\
                                & \textbf{FR04} & \noindent\parbox[c]{\hsize}{
                                                  Extrahierte Interaktionen müssen für den
                                                  Zugriff späterer Gruppen persistiert werden} \\
        \textbf{Muss-Kriterien} & \textbf{FR05} & \noindent\parbox[c]{\hsize}{
                                                  Spätere Gruppen müssen auf die persistierten
                                                  Nachrichten zugreifen können} \\
                                & \textbf{FR06} & \noindent\parbox[c]{\hsize}{
                                                  Gruppe 1 muss die Möglichkeit haben, uns über
                                                  die Verfügbarkeit neuer Daten zu
                                                  benachrichtigen} \\
                                & \textbf{FR07} & \noindent\parbox[c]{\hsize}{
                                                  Gruppe 3 muss von uns benachrichtigt werden,
                                                  wenn neue Daten zur Verfügung stehen} \\
                                & \textbf{FR08} & \noindent\parbox[c]{\hsize}{
                                                  Parteien und Abgeordnete müssen über Sitzungen
                                                  hinweg eindeutig zuordenbar sein} \\
                                & \textbf{FR09} & \noindent\parbox[c]{\hsize}{
                                                  Daten von Gruppe 1 müssen aus deren MongoDB
                                                  ausgelesen werden können} \\
        \hline

        \textbf{Soll-Kriterien} & \textbf{FR10} & \noindent\parbox[c]{\hsize}{
                                                  Das Open Data Ausgabeformat des Bundestags soll
                                                  eingelesen werden können} \\
        \hline

        \textbf{Kann-Kriterien} & \textbf{FR11} & \noindent\parbox[c]{\hsize}{
                                                  Interaktionen innerhalb der Redebeiträge können
                                                  extrahiert werden} \\

    \end{tabularx}
    \label{tab:03_requirements}
\end{table}

\subsection{Wahl des Kommunikationsmodell}

\section{Umsetzung}\label{sec:03_02_umsetzung}


\subsection{Communication Model Extractor (CME)}

\subsection{Eingabeformate}
Wie in den Anforderungen FR02 und FR10 definiert, muss als Eingabeformat
einerseits das JSON Format der Gruppe 1 und andererseits soll auch das
Open Data XML Format des Bundestags unterstützt werden. So wurden
spezielle Parser für die jeweiligen Formate entwickelt, deren Ausgabe gleich
formatiert ist und somit für die nächste Instanz eine einheitliche
Schnittstelle bildet. Dieses Ausgabeformat enthält dabei neben Metadaten zur
Sitzung selbst, wie z.~B. den Sitzungsstartzeitpunkt oder die Sitzungsnummer,
sogenannte InteractionCandidates. Diese gruppieren, wie der Name bereits
nahelegt, Daten, welche eventuelle für die Auswertung relevante Interaktion
enthalten. So gruppiert ein InteractionCandidate den Redner (Speaker),
einen Absatz von dessen Rede (Paragraph) und den eventuell darauf folgenden
Kommentarblock (Comment).

\subsection{Methodik der Kommunikations-Analyse}
Die zu analysierenden Daten (InteractionCandidates) bieten zwei verschiedene
Kommunikationstypen als Datengrundlage für die Erkennung von Interaktion:

\begin{enumerate}
    \item \textbf{Kommentare (FR03)}: werden von den Redeteilen gesondert
        dargestellt und mit zusätzlicher Information bezüglich der
        kommunikativen Einordnung des Beitrags versehen.
    \item \textbf{Redeteile (FR10)}: werden in den Protokollen als
        Paragraphen oder sinngemäß als Absätze aufgeführt.
\end{enumerate}

Aufgrund der Unterschiedlichkeit müssen beide Typen gesondert analysiert werden.

\subsubsection{Kommentare}

Kommentare werden nach rigiden strukturellen Regeln in den Protokollen der
Bundestagssitzungen erfasst und liefern Informationen zum Sender des
Kommentars. Dies kann eine oder mehrere Fraktionen oder eine oder mehrere an
der Sitzung teilnehmende Personen sein. Dabei lassen sich Kommentare in drei
Kategorien unterteilen:

\begin{itemize}
    \item \textbf{Publikumsresonanzen}: Z.~B. Beifall oder Entrüstung des
        Senders, bei der eine genaue Zitierung durch die Stenographen nicht
        möglich ist. Hier folgen Sender durch Präpositionen abgetrennt auf den
        Kommentarinhalt. Bsp.: \enquote{Beifall bei der CDU}; \enquote{Zurufe
        von der Linken und Grünen}
    \item \textbf{Zitate}: Hier wird der Wortlaut eines Kommentars
        wiedergegeben. Bei diesen Kommentaren wird die kommentierende Person
        dem Wortlaut vorangestellt und durch einen Doppelpunkt abgetrennt.
        Bsp.: \enquote{Dr. Klaus Hermann: Ganz toll gemacht}; \enquote{Dr.
        Alexander Gauland: Schwachsinn}
    \item \textbf{Beobachtungen der Protokollanten}: Spezielle Situationen in
        denen ein oder mehrere Sitzungsteilnehmer nennenswerte Handlungen
        vornehmen. Auch hier wird in der Regel der Sender der Nachricht
        vorangestellt, wenngleich keine Trennung durch einen Doppelpunkt
        erfolgt. Bsp.: \enquote{Manfred von Kuchenhausen verlässt die Sitzung}
\end{itemize}

Je nach erkannter Kommentarform kommen also verschiedene strukturelle
Analyse-Vorgänge für die Feststellung von Sendern zum Einsatz.

\paragraph{Kommentare mit abweichender Struktur}
\subsubsection{Redeteile}
Redeteile besitzen weniger Struktur als Kommentare, da sie nicht den Regeln
der Stenographen unterliegen, sondern des individuellen Redners. So muss der
Empfänger einer Aussage erst ermittelt werden.

Redeteile werden daher auch nur als Interaktionen gewertet, sobald Fraktionen
oder Mitglieder des Bundestags als mögliche Interaktionsempfänger im Fließtext
erkannt werden.

Die Erkennung von MDBs erfolgt über die Feststellung bekannter, eindeutiger
Nachnamen, denen eine formale Anrede (Herr/Hr., Frau/Fr., Kollege, Doktor)
vorangeht. Parteien werden anhand von ihren Namen bzw. deren Abkürzungen,
beziehungsweise durch gängige aber eindeutige Kurzformen identifiziert. Die
festgestellte Person oder Partei wird anschließend als Empfänger der
Nachricht behandelt, während die sprechende Person, also der durch den
Sitzungspräsidenten zuletzt eingeleitete Redner, die Rolle des Absenders
einnimmt.

Die Empfängerermittlung bei Redeanteilen verbleibt in der entstandenen
Implementierung in einem recht rudimentären Zustand, da viele der inhärenten
Probleme des Ansatzes nicht behandelt werden konnten. So kann etwa die
Unterscheidung zwischen mehreren MDBs mit gleichen Nachnamen oder die
kolloquiale Verwendungen von Farben (\enquote{Rot} für SPD oder die Linke) oder
Begriffen (\enquote{Liberale} stellvertretend für die FDP) als Substitut für
Fraktionen nicht behandelt werden. Um diese Problematiken zu umgehen
werden in der Paragraphen-Analyse nur solche Interaktionen gewertet,
welche sich an eindeutig identifizierbare MDBs oder Fraktionen richten.
Nachnamen wie \enquote{Müller}, die mehrere MDBs bezeichnen können, aber auch in
anderen Sachverhalte eingesetzte Begriffe, wie \enquote{Union}, werden somit aus der
Schlüsselwort-Liste verbannt, mit der die Redeabschnitte abgetastet werden.
Dadurch wird eine unbestimmt große Menge an Interaktionen nicht registriert,
was mit Sicherheit eine Verzerrung der Projektergebnisse, aber gleichzeitig
auch ein wesentliches Verbesserungspotential für die mögliche Fortsetzungen
des Projekts darstellt. Gleichzeitig sollte nicht außer Acht gelassen werden,
dass Interaktionen in Redeteilen nach unseren Messungen einen recht geringen
Anteil der Interaktionen im Bundestag ausmachen. Die Steigerung der Anzahl
erkannter Interaktionen durch unsere Methode der Redebeitrag-Analyse beträgt
für die bisherigen Protokolle der 19. Wahlperiode lediglich 8\% gegenüber der
alleinigen Betrachtung der Kommentare.

\subsection{Ermittlung von teilnehmenden Personen}
Personen müssen über Sitzungen hinweg eindeutig identifizierbar sein, damit
nachfolgende Gruppen Informationen zu Parteizugehörigkeit und Namen korrekt
zuweisen können (FR08). Personen werden in der \enquote{mdb} Collection der MongoDB
gesammelt. Initialisiert werden kann diese Liste mithilfe der Stammdaten der
Bundestagsabgeordneten des Bundestags. Diese als XML zur Verfügung gestellte
Liste enthält alle relevanten Informationen, die bei der Identifizierung
helfen (Namen und Namensänderungen, Parteizugehörigkeit mit von-bis
Datumsangabe, Titel etc.).

Da die Stammdaten unvollständig sind, auch weil Gastredner Reden halten oder
Bundestagsabgeordnete in den Stammdaten fehlen, müssen weitere Redner während
der Analyse hinzugefügt werden. Gerade in den Kommentaren können meist Vor-
und Nachnamen extrahiert werden, um ein Abgleich mit den MDB-Daten
vorzunehmen. Leider werden Namen oft inkonsistent geschrieben (z.~B. wird
der zweite Vorname manchmal ausgeschrieben, abgekürzt oder weggelassen) oder
die Stenographen vertippen sich. Auch wird das Format der Kommentarsektion
nicht immer eingehalten.

Weil wir fehlende Redner und Rechtschreibfehler nicht unterscheiden können,
gelangen doppelte Einträge in die Datenbank. Wir haben Mechanismen eingebaut,
um solche Fehler zu erkennen, können Sie aber nicht ganz ausschließen.

\subsection{Infrastruktur Setup mit Docker}
Um ein einfaches Deployment sicherzustellen, wurde sowohl die mongodb als
auch der Service selbst mit Docker bzw. docker-compose abstrahiert.

Die Standardkonfiguration des mongodb Containers wurde lediglich um
Berechtigungen erweitert.

\subsubsection{Datensicherheitsvorfall}

Während der Projektimplementierung wurde der MongoDB Server, auf dem die
Kommunikationsdaten abgelegt wurden, aufgrund von Fehlkonfigurationen
kompromittiert.

Die Firewall wurde zwar korrekt konfiguriert, der Docker-Daemon öffnet mit
seinen Root-Rechten Container jedoch standardmäßig nach außen. Außerdem wurde
kein Benutzername oder Passwort für die Datenbank gesetzt. Somit konnte ein
Angreifer den offenen MongoDB-Port finden, die Datenbank löschen und eine
Erpressernachricht hinterlassen - ein klassischer Ransomware-Angriff.

Weil keine der Daten prägnant waren (da sie ohnehin im Netz frei verfügbar
waren), konnte über den Verlust hinweg gesehen werden.

Der zuständige Laboringenieur wurde kontaktiert und der Server zurückgesetzt.
Nach einer Analyse der Fehler wurden Zugangsdaten für die Datenbank angelegt
sowie Docker umkonfiguriert.

Danach wurden die Maßnahmen getestet, um sicherzustellen, dass nur noch
Nutzer mit SSH und Datenbank-Zugangsdaten Zugriff haben.

Zukünftig wäre es sinnvoll den Docker-Daemon als Nicht-Root-User laufen zu
lassen, um die Anwendung weiter abzusichern.

\subsection{Schnittstelle für Zugriff auf den Communication Model Extractor]}

Um den gewünschten Workflow einer Pipeline zu ermöglichen, entschieden wir
uns eine API anzubieten, wodurch die Kommunikation zur vorherigen und
folgenden Gruppe realisiert wurde.

\begin{figure}[ht]
    \begin{center}
        \includegraphics[width=\textwidth]{example-image-a}
    \end{center}
    \caption{TODO: write me}
    \label{fig:02_api_call_flow}
\end{figure}

Es wurde sich mit Gruppe 1 darauf geeinigt, dass der CME einen Endpunkt
anbietet, um die Information zu erhalten, wenn neue Protokolle aus dem
Bundestag in deren Datenbank kreiert wurden. Zusätzlich bekamen wir
Zugangsdaten, um auf deren Datenbank zugreifen zu können. So werden uns die
Protokoll-ID's an \url{http://infosys2.f4.htw-berlin.de:9001/cme/data/} gesendet
und der CME holt sich die gewünschten Protokolle zu einem späteren Zeitpunkt
aus deren Datenbank. Danach kann die Auswertung geschehen.

Um der darauffolgenden Gruppe Bescheid zu geben, wird nach dem Prozess der
Auswertung ebenfalls eine HTTP Anfrage gesendet, in der über die neuen
Protokoll-ID's informiert wird. Unsere API bietet dafür verschiedene Endpunkte
an, um an alle notwendigen Informationen zu gelangen.

Diese Endpunkte sind alle in einem Endpunkt festgehalten, die die API
dokumentiert und hier abgerufen werden kann:
\url{http://infosys2.f4.htw-berlin.de:9001/cme/doc/docs}

Demnach enthält der CME folgende Endpunkte:
\begin{itemize}
    \item /cme/data/sessions/ - damit wird eine Liste aller existierenden
        Sitzungen zurückgegeben
    \item /cme/data/session/\{session\_id\} - um eine bestimmte Sitzung mit den
        ermittelten Interaktionen zu erlangen
    \item /cme/data/mdb - nützlich, um nach Mitgliedern des Bundestags zu
        suchen, mithilfe der Query-Parameter mdb\_id, speaker\_id, forename
        \& surname kann gefiltert werden
    \item /cme/data/faction - damit können alle existierenden Fraktionen mit
        jeweiliger ID abgerufen werden
\end{itemize}

Der Server ist nur über das HTW-Netzwerk erreichbar. Zusätzlich ist die API mit
Basic Authentication geschützt. Dadurch kann nur auf die Daten zugegriffen
werden, wenn sich mit Username und Passwort authentifiziert wird.

Es wurden drei Clients, sprich User, angelegt, die auf die API zugreifen
dürfen. Je einer für die anderen Gruppen crawler\_client \& sentiment\_client und
cme\_admin für unser Team, um Testen zu können und bei Bugs der Ursache auf den
Grund gehen zu können.

\section{Fazit}\label{sec:03_05_fazit}

Abschließend konnten die meisten Anforderungen erfolgreich umgesetzt werden.

Der Service bietet Schnittstellen zur vorherigen und nachfolgenden Gruppe und
sowie automatisierte Verarbeitungs-Mechanismen (notify). Auch die frühzeitige
Bereitstellung einer akzeptierten Kommunikations-Modellierung auf Basis der XML-
Rohdaten konnte realisiert werden. Der Großteil des Datenbestands wird
analysiert, sodass sowohl Kommentare als auch Redeteile auf Kommunikationen
überprüft werden.

Allerdings verwirft der Service, aufgrund von Inkonsistenzen in der Syntax und
Rechtschreibung sowie aufgrund von fehlenden kontextbasierten Analyse-
Mechanismen, potentielle Interaktionen. Während in Kommentaren primär durch
Abweichungen von Protokoll-Syntax und Rechtschreibfehler Probleme entstanden,
so fehlte in den Redeteilen oft die nötige Information um die Empfänger von
Interaktionen eindeutig zu bestimmen. Daher mussten Interaktionen verworfen
werden und teils wurden in Kommentaren inkorrekt genannte Personen als
Personenduplikate in den Datenbestand aufgenommen.

Daher wurde das Ziel der Gruppe “Kommunikationsmodell” in Funktionalität und
Integration zwar erreicht, allerdings wird deutliches Verbesserungspotential
für die Extraktion von noch mehr Interaktionen und die Eliminierung von
Artefakten gesehen

Die ausgelieferte Deployment-Strategie und die Modulbeschreibungen in dieser
Dokumentation sollten bei der Weiterentwicklung des Projekts hilfreich sein.