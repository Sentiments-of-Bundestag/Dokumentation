\documentclass[a4paper,12pt,twoside]{book} 
\usepackage[utf8]{inputenc}

\usepackage{listings}
\usepackage{xcolor}

%New colors defined below
\definecolor{codegreen}{rgb}{0,0.6,0}
\definecolor{codegray}{rgb}{0.5,0.5,0.5}
\definecolor{codepurple}{rgb}{0.58,0,0.82}
\definecolor{backcolour}{rgb}{0.95,0.95,0.92}

%Code listing style named "mystyle"
\lstdefinestyle{mystyle}{
  backgroundcolor=\color{backcolour},   commentstyle=\color{codegreen},
  keywordstyle=\color{magenta},
  numberstyle=\tiny\color{codegray},
  stringstyle=\color{codepurple},
  basicstyle=\ttfamily\footnotesize,
  breakatwhitespace=false,         
  breaklines=true,                 
  captionpos=b,                    
  keepspaces=true,                 
  numbers=left,                    
  numbersep=5pt,                  
  showspaces=false,                
  showstringspaces=false,
  showtabs=false,                  
  tabsize=2
}

%"mystyle" code listing set
\lstset{style=mystyle}

% standard incantations
\usepackage[T1]{fontenc}
\usepackage[utf8]{inputenc}
\usepackage{lmodern}
\usepackage[german]{babel}
\usepackage{csquotes}

% clickable links in the PDF
\usepackage{hyperref}
\usepackage{float}
% bibliography
\usepackage[sorting=none]{biblatex}
\addbibresource{literatur.bib}

% glossary
\usepackage[xindy]{glossaries} 
\newglossaryentry{souveraenitaet}{%
  name={Souveränität},%
  description={Der Begriff Souveränität, auch „Staatshoheit“, wird im innerstaatlichen Recht und in der politischen Theorie verwendet, um die oberste Kompetenz zur Machtausübung im Innern eines Staates zu bezeichnen\cite{grundgesetzSouv}.}}

\newglossaryentry{kks}{%
  name={KKS},%
  description={künstliche Währung zum Ausgleich von Preis­niveau-Unterschieden zwischen den Mitgliedstaaten der Europäischen Union; ein Kaufkraftstandard (KKS) entspricht der durchschnittlichen Kaufkraft eines Euro in der Europäischen Union (EU-27)}}

\newacronym{eu}{EU}{Europäische Union}

% generic
\newacronym{api}{API}{Application Programming Interface}
\newacronym{ip}{IP}{Internet Protocol}
\newacronym{ssh}{SSH}{Secure Shell}
\newacronym{cli}{CLI}{Command Line Interface}
\newacronym{htw}{HTW}{Hochschule für Technik und Wirtschaft}
\newacronym{xml}{XML}{Extensible Markup Language}
\newacronym{json}{JSON}{JavaScript Object Notation}
\newacronym{sob}{SoB}{Sentiments of Bundestag}
\newacronym{rest}{REST}{Representational State Transfer}
\newacronym[plural=MDBs, firstplural=Mitglieder des Deutschen Bundestages (MDBs)]{mdb}{MDB}{Mitglied des Deutschen Bundestages}

% group 2
\newacronym{cme}{CME}{Communication Model Extractor}


\makeglossaries

% add literatur to toc
\usepackage[nottoc]{tocbibind}

% graphics and images
\usepackage{graphicx}
\usepackage{subfigure}
\usepackage{wrapfig}

% color packages
\usepackage{color, colortbl}
\definecolor{Gray}{RGB}{220,220,220}
\definecolor{EUBlue}{RGB}{45,172,227}
\definecolor{White}{RGB}{255,255,255}
\usepackage[first=0,last=9]{lcg}
\newcommand{\ra}{\rand0.\arabic{rand}}

% multirow table
\usepackage{multirow}

% footnote package
\usepackage{tablefootnote}

\lstset{
  captionpos=b,
  commentstyle=\color{UStuttDarkGreen},
  frame=single,	                   % adds a frame around the code
  keepspaces=true,
  %keywordstyle=\color{UStuttDarkBlue},
  showspaces=false,
  showstringspaces=false,          % underline spaces within strings only
  showtabs=false,
  stringstyle=\color{UStuttDarkBlue},
  tabsize=2
}

% --------------------------------------------------------------------
% Definitions of title informations
% --------------------------------------------------------------------
\newcommand{\HRule}[1]{\rule{\linewidth}{#1}}

\makeatletter
\def\printtitle{	
    {\centering \@title\par}}
\makeatother			

\makeatletter
\def\printauthor{
    {\centering \large \@author}}
\makeatother

% --------------------------------------------------------------------
% Config Title & Author
% --------------------------------------------------------------------
\title{
\HRule{0.5pt} \\
\LARGE \textbf{\uppercase{Sentiments of Bundestag}}
\HRule{2pt} \\ [0.5cm]
\normalsize \textsc{Graph-basiertes Informationssystem zur Analyse sozialer Interaktion im Deutschen Bundestag}
\\[2.0cm]
\normalsize \today
}

\author{
\normalsize Betreut von
\normalsize Prof. Dr. Thomas Hoppe\\
\normalsize Informationssysteme\\
\normalsize M.Sc. Angewandte Informatik\\
\normalsize Hochschule für Technik und Wirtschaft\\
\normalsize Treskowallee 8, 10318 Berlin, Deutschland\\
}

\begin{document}
% ------------------------------------------------------------------------------
% Maketitle
% ------------------------------------------------------------------------------
\thispagestyle{empty}
\printtitle
  	\vfill
\begin{figure}[H]
    \centering
    \includegraphics[width=200px, keepaspectratio]{logos/bundestag.png}
\end{figure}
  	\vfill
\printauthor		
\newpage

\pagenumbering{roman}
\setcounter{page}{3}

\setcounter{tocdepth}{2}
\tableofcontents

\listoffigures

\listoftables

\pagenumbering{arabic}
\setcounter{page}{6}

\chapter{Vorwort}
\section{Einleitung}\label{sec:08_01_einleitung}
\section{Aufbau der Lösung}\label{sec:01_02_aufbauLoesung}
Das Projekt zur Entwicklung eines graphbasierten Informationssystem für die Analyse sozialer Interaktionen im Deutschen Bundestag, welches \glqq Sentiments Of Bundestag\grqq{} genannt wurde, besteht, wie bereits erwähnt, aus sieben Teilprojekten. Im folgenden Abschnitt wird grob auf die Thematiken der einzelnen Gruppen eingegangen.

\begin{figure}[H]
    \centering
    \includegraphics[width=\textwidth]{images/01-Einleitung/SentimentOfBundestag.png}
    \caption{Aufbau der Lösung}
    \label{fig:aufbauderLösungSOB}
\end{figure}

Die in \autoref{fig:aufbauderLösungSOB} dargestellten Teilprojekte sind hier nun detaillierter aufgelistet:
\begin{itemize}
    \item \textbf{Crawler}: Scannt regelmäßig die Open Data Webseite des Bundestags, sucht, parst und speichern neue Protokollen sowie Stammdaten der Abgeordneten in seiner No-Sql-Datenbank. Ziel ist es hier sicherzustellen, das die DB immer auf dem neusten Stand bleibt
    \item \textbf{Kommunikationsmodell}: Analysiert und erstellt aus den
      Protokollen von Gruppe 1 ein Kommunikationsmodell, welches die möglichen
      Interaktionen im Bundestag abbildet
    \item \textbf{Sentiment-Analyse}: Errechnet die Stimmung der von Gruppe 2 identifizierten Interaktionen und stellt Daten für Gruppe 3 und 4 bereit
    \item \textbf{Interaktion zwischen Abgeordneten}: Aus dem Kommunikationsmodell von Gruppe 2 und die Stimmungsanalyse von Gruppe 3 werden hier Interaktionen zwischen einzelnen Personen (Abgeordneten, Präsident, Gäste, etc.) identifiziert. Erstellt wird wird daraus ein gewichteter Sentiment-Graph zwischen Abgeordneten mit positiven/negativen Gewichtungen
    \item \textbf{Interaktion zwischen Fraktionen}: Aus dem Kommunikationsmodell von Gruppe 2 und den Sentiment-Graph zwischen Personen von Gruppe 4 werden hier Interaktionen zwischen Gruppen von Personen analysiert und in einen Sentiment-Graph zwischen Parteien. Der besteht aus einer Aggregation der Abgeordnetensentiments zu gewichteten Sentiment-Graph der Parteien (Fraktionen, Gruppen, etc.)
    \item \textbf{Graphauswertung}: Die Sentiment-Graphen von den Gruppen 4 und 5 werden hier anhand verschiedener Auswertungsmethoden analysiert und die Ergebnisse davon der nächsten Gruppe (Benutzeroberfläche) zur Verfügung gestellt
    \item \textbf{Benutzeroberfläche}: Ziel ist hier die Realisierung einer interaktiven Benutzeroberfläche zur Darstellung der Ergebnisse
\end{itemize}

Die einzelnen Teilprojekte werden in den nächsten Kapiteln von den jeweiligen Gruppenmitgliedern genauer erläutert.


\chapter{Crawler}
\thispagestyle{empty}
\begin{table}[ht]
\caption{Gruppe 1 (Crawler) - Arbeitsaufteilung}
\label{tab:zeittafelEU}
\centering
\begin{tabular}{m{6em}|m{12em}|m{14em}}
\hline
\rowcolor{Gray}
Aufgabe &Gruppenmietglieder &Anteil\\
\hline
\hline
\rowcolor{White}
Crawl-Manager& Boris Foko Kouti & Scheduler, TaskManager, Rest-API\\
\hline
\hline
\rowcolor{White}
& Marlon Daniel Kohlberger & XML-Parser\\
\cline{2-3}
\rowcolor{White}
\multirow{-2}{*}{Crawl-Utilities}& Boris Foko Kouti & Page Loader, Page Analyser\\
\hline
\hline
\rowcolor{White}
& Marlon Daniel Kohlberger & DB-Modell\\
\cline{2-3}
\rowcolor{White}
\multirow{-2}{*}{Datenbank}& Boris Foko Kouti & DB-Manager\\
\hline
\hline
\cline{2-3}
\rowcolor{White}
& Arnauld Feussi & Erste Vorlage \\
\cline{2-3}
\rowcolor{White}
& Marlon Daniel Kohlberger & DB-Modell, Parser, Refactoring \\
\cline{2-3}
\rowcolor{White}
\multirow{-2}{*}{Dokumenation}& Boris Foko Kouti & Einleitung, Crawl-Mechanismus und Bereitstellung\\
\hline
\hline
\cline{2-3}
\rowcolor{White}
& Arnauld Feussi & Server Aufsetzung\\
\cline{2-3}
\rowcolor{White}
& Marlon Daniel Kohlberger & Sicherheit Config und Firewall Config\\
\cline{2-3}
\rowcolor{White}
\multirow{-2}{*}{Bereitstellung}& Boris Foko Kouti & Crawler Service und MongoDB\\
\hline
\end{tabular}
\end{table}
\section{Einleitung}\label{sec:08_01_einleitung}
\section{Anforderungen und Rahmenbedingungen}\label{sec:02_02_anforderungen_rahmen}
Der zu entwickelnden Crawler soll in bestimmten Rahmenbedingungen einige Anforderungen erfüllen. Diese sowie die technischen und organisatorischen Rahmenbedingungen dazu werden in diesem Abschnitt aufgelistet. 
\subsection{Anforderungen an dem Crawler}\label{subsec:02_02_anforderungen}
Die Anforderungen an dem Crawler sind in~\cite{Hoppe2020Uebung} zusammengefasst und sehen vor, dass der Crawler kontinuierlich laufen sollte und dabei nach folgenden Dateien suchen:
\begin{itemize}
  \item \textbf{Protokolle}: Ein Protokoll ist eine XML-Datei, die den gesamten Ablauf einer Plenarsitzung (Inhaltsverzeichnis, Sitzungsverlauf mit Sitzungsbeginn, Tagesordnungspunkte, Sitzungsende, Anlagen, Rednerliste und  gewisse Konfigs) beinhaltet. Da sich die Formatierung der Protokollen in den Jahren stets verbessert hat, soll hier auf die Legislaturperioden geachtet werden:
  \begin{itemize}
    \item 19. Legislaturperiode: diese sind sehr gut strukturiert und leicht zu parsen
    \item 18. Legislaturperiode: schlechter formatiert als die 19. Legislaturperiode. Hierfür wird vom Prof. Dr. Thomas Hoppe eine mit besser formatierte Version zur Verfügung gestellt
  \end{itemize}
  \item \textbf{Stammdaten der Abgeordneten}: Ist eine XML-Datei, die Informationen über Abgeordneten seit 1949 sammelt
  \item \textit{Optional} \textbf{Namentliche Abstimmungen}: Liste aller namentlichen Abstimmungen im Bundestag seit dem 18.12.2009 in PDF-Format und seit dem 18.10.2012 auch in XLSX-Format~\cite{NamentlicheAbstimmung20}
  \item Es muss sichergestellt sein, dass alle Daten immer auf dem neusten Stand sind. Dafür soll die Frequenz des Crawlers sich an dem Sitzungskalender des Bundestags~\cite{Sitzungskalender2021} orientieren
\end{itemize}

\subsection{Rahmenbedingungen}\label{subsec:02_02_rahmenbedingungen}
\subsubsection{Organisatorische Rahmenbedingungen}
Für die Planung des gesamten Projekts ist in einem zwei Wochen-Takt ein Plenum für den Brainstorming und eventuelle Teams-übergreifende Abstimmungen festgelegt worden. Im Team-Crawler (Gruppe 1) ist folgende Planung abgemacht worden:
\begin{itemize}
    \item Ziel bis zum 13.11.2020: Testdaten für andere Gruppen, erster lauffähiger Prototyp
    \item Ziel bis zum 27.11.2020: Vorläufiger Release und Integration-Test mit anderen Gruppen (Kommunikationsmodell)
    \item Ziel bis zum 04.12.2020: finale Version und Anfang der Dokumentation
    \item Ziel bis zum 25.02.2021: Fertigstellung der Dokumentation und Abgabe
\end{itemize}
Neben den organisatorischen Rahmenbedingungen sind technische Rahmenbedingungen zu betrachten bzw. sind Fehler (Probleme) zu vermeiden (zu lösen).
\subsubsection{Technische Rahmenbedingungen: Probleme beim Crawlen}\label{subsubsec:techAnforderungen}
\begin{itemize}
    \item Sperrung durch Server-Administrator (Server-Regel). Es soll vermieden werden, dass der ausführende Rechner wegen zu häufige Abfragen vom Server-Admin gesperrt wird und nur noch ein 500-Fehlercode erhält
    \item Ajax basierende Inhalte. Die Open Data Webseite verwendet Ajax für die Bereitstellung der Daten und lädt diese im Hintergrund. Ein einfacher Download der Seite genug da nicht. Außerdem nutzen die meisten Links, hinter denen Dateien stehen, Javascript (es ist z. B. der Fall bei Slides und verborgenen Regionen). Dies muss entsprechend gehandelt werden
    \item Mongo-DB-Konfiguration und Zurverfügungstellung der Daten für die Gruppe Kommunikationsmodell 
    %\item Das robots.txt Protokoll
\end{itemize}
\section{Lösung und Konzepte}\label{sec:02_03_loesung_konzept}
\subsection{Standard Aufbau eines Crawlers}
\begin{wrapfigure}{R}{0.4\textwidth}
  \centering
    \includegraphics[width=0.4\textwidth]{images/02-Crawler/Crawler-Standard-Crawler.png}
  \caption{\label{fig:standardCrawler}Crawler: Standard Aufbau}
\end{wrapfigure}
Ein Crawler besteht in der Regel aus zwei wichtigen Teilen: ein Downloader und ein Scheduler. Der Downloader ist mit dem Webserver verbunden und lädt die Webseite(n) sowie alle erwünschten Dateien herunter. Der Scheduler ist, wie der Name schon sagt, für die Planung zuständig und erteilt dem Downloader URLs nach internen Priorisierung-Mechanismen (als Liste FIFO und als Graph mit DFS und BFS ~\cite{ThomasAlgorithms2009}--\cite{DonaldKnuth1998}). 
\subsection{Crawler Mechanismus}
Für unseren speziellen Fall, wird die Aufgaben von dem Scheduler vom Crawl-Manager übernommen, der sowohl für die Planung als auch für die Steuerung (Start, Pause, Stopp, Zustand) laufender Crawl-Aufgaben zuständig ist. Der Crawl-Manager erzeugt für jede URL ein Thread (Page-Crawler), dessen Aufgabe darin besteht die Seite herunterzuladen, mit seinem Page-Analyser nach relevanten Inhalten in der Seite zu suchen (URLs, Dateien, Action-Links: Javascript, etc.) und bei Bedarf weitere parallele Sub-Threads für den Download, das Parsen und die Speicherung der relevanten Dateiinhalte (Protokollen, Stammdaten, weitere Metadaten). Dieser Mechanismus wird durch die Abb.~\ref{fig:funktionsprinzipCrawler} und die Abb.~\ref{fig:crawlEinerUrl} veranschaulicht.

\begin{figure}[H]
    \centering
    \includegraphics[width=5in]{images/02-Crawler/Crawler-Process-Diagram (Multithreading).png}
    \caption{Funktionsprinzip des unseres Crawlers}
    \label{fig:funktionsprinzipCrawler}
\end{figure}

\begin{figure}[H]
    \centering
    \includegraphics[width=5.5in]{images/02-Crawler/Crawler-Process-Diagram (Page-Crawler).png}
    \caption{Page-Crawler für eine URL}
    \label{fig:crawlEinerUrl}
\end{figure}
\noindent
Ein Crawl auf einer Seite (durch den Page-Crawl) kann in einer durch den Scheduler (Crawl-Manager) regelmäßig gemacht werden, in dem dieser nach einer bestimmten Zeit einen neuen Page-Crawl-Thread startet. Die Frequenz der Generierung dieses Threads orientiert sich an dem Sitzungskalender des Bundestags~\cite{Sitzungskalender2021}. Dieser Kalender sieht vor, dass Sitzungen an bestimmten Wochentagen zwei bis vier Wochen pro Monat (abgesehen von August: Ferien) stattfinden. Darauf basierend ist die Entscheidung getroffen worden, die Frequenz auf einmal pro Tag von Montag bis Freitag (mit der Möglichkeit bei Bedarf den Crawler in der Ferienzeit zu stoppen) festzulegen. So lässt sich eine Sperrung wegen zu häufiger Abfragen verhindern. Bei Manchen Server (durch Regeln oder Server-Admin) kann aber ein sich wiederholender Abfrage-Muster (wie eine Abfrage jeden Tag um dieselbe Uhrzeit) auch zu einer Sperrung der IP (Rechner) führen. Aus diesem Grund wird zusätzlich zur Frequenz ein Zufallsfaktor verwendet. Ein Page-Crawl-Thread wird zwar von Montag bis Freitag um 23Uhr durch den Crawl-Manager gestartet, allerdings wartet dieser Thread ein zufällige Zeit $t$ (mit $10 \leq t \leq 7200 $) bis er den tatsächlichen Crawl durchführt. Während und nach dem Crawl-Prozess werden Daten (und Dateien) manipuliert, analysiert und gespeichert. Diese Vorgänge sowie die dafür verwendeten Datenstrukturen werden im nächsten Abschnitt beleuchtet. 

\subsection{Daten-Parser und Database-Modell}
- Parser für die 19. Legislaturperiode\\
- Parser für die 18. Legislaturperiode\\
- ER-Diagramm und Beschreibung des DB-Modells

\subsection{Gesamter Aufbau der Lösung}
Die gesamte Lösung besteht aus vier Komponenten und einer Datenbank, wie auf die Abb.~\ref{fig:crawlerKompoenenten} dargestellt. 
\begin{itemize}
    \item \textbf{Crawl-Manager}: Taskmanager und Scheduler
    \item \textbf{Crawl-Utilities}: Liefert die für den Crawl-Prozess nötigen Funktionalitäten (Page-Fetcher, Page-Analyser, Downloader, Data-Parser)
    \item \textbf{DB-Manager}: Verwaltet den Zugriff auf die Datenbank 
    \item \textbf{Rest-ServiceProvider}: Rest-API für die Steuerung des Crawl-Manager und den eingeschränkten Zugriff auf die DB-Daten
    \item \textbf{Datenbank}: NoSql (MongoDB) Datenbank für die Sicherung der Daten
\end{itemize}

\begin{figure}[H]
    \centering
    \includegraphics[width=5in]{images/02-Crawler/Crawler-Component-Diagram.png}
    \caption{Crawler: Komponentendiagramm}
    \label{fig:crawlerKompoenenten}
\end{figure}
\noindent
Am Ende eines Crawl-Prozess, bei dem neue Protokolle oder Stammdaten geladen worden sind, wird die Gruppe-Kommunikationsmodell die Liste aller Protokollen sowie Stammdaten über deren Rest-API~\cite{Cme2021} als Json-Datei mitgeteilt. Die Gruppe-Kommunikationsmodell kann dann asynchron die entsprechenden Dateien aus unseren Datenbank holen.
\section{Implementierung und Bereitstellung}\label{sec:02_04_implementierung_bereitstellung}

\subsection{Implementierung des Crawlers}
Für die Umsetzung des Crawlers wurde die Entscheidung getroffen eine Webanwendung zu implementieren. Für dieses Vorhaben wurde Spring Boot~\cite{SpringBoot242} gewählt, welches ein auf der Programmiersprache Java basierendes Framework für die Entwicklung von Webanwendungen ist. Mithilfe des Frameworks lassen sich Konfigurationen für REST-APIs, Datenbanktreiber, etc. über den Spring-Initializer vereinfachen. Außerdem verfügt Spring Boot über einen eigenen Task-Scheduler der für den Crawl-Manager verwendet werden kann. Die finale Lösung beinhaltet vier Packages: \\\textbf{crawler} mit den Funktionalitäten für den Crawl-Prozess und das Parsen, \textbf{models} und \textbf{repositories} für den DB-Manager. Des Weiteren das Package \textbf{web} mit den Controllern für die Rest-Schnittstelle und den Services für den Crawl-Manager. Das gesamte Projekt-Verzeichnis inklusive des Codes ist auf dem Github-Repository \textit{Sentiments-of-Bundestag/Crawler}~\cite{Crawler2021} zu finden. Für den Download bestimmter Informationen ist es nötig auf Javascript basierende Inhalte auszuführen. Um dies durchzuführen wird die \\\textbf{HtmlUnit}-Bibliothek~\cite{HtmlUnit2021} genutzt. (siehe ~\ref{subsubsec:techAnforderungen}).

\subsection{Bereitstellung der Lösung}
Der Crawler wurde auf dem Virtual-Server der Gruppe 1 (141.45.146.161) an der HTW als Ubuntu-Service bereitgestellt. Dieser ist aus Sicherheitsgründen nur im HTW-Netzwerk sichtbar. Der Crawler-Service liefert eine REST-Schnittstelle, die jedoch aus Sicherheitsgründen nur auf dem infosys1 Rechner zugänglich ist. Die Konfiguration des Crawler-Services, die unter \textit{/etc/systemd/system/crawler.service} angelegt wird, sieht wie folgt aus (Start des Crawlers: systemctl enable crawler):
\begin{lstlisting}
[Unit]
Description=Crawler App
User=local
[Service]
ExecStart=/usr/bin/java -jar ../Crawler-1.0-SNAPSHOT.jar
Restart=always
SyslogIdentifier=crawler
[Install]
WantedBy=multi-user.target
\end{lstlisting}
Da für den Crawler eine MongoDB benötigt wird, soll diese auch bei der Bereitstellung konfiguriert werden. Dabei muss sichergestellt werden, dass der Zugriff auf die Datenbank nur mit den entsprechenden Zugangsdaten erfolgt. Somit ist zu beachten, dass nach dem Hinzufügen eines neuen Admin-Users die Security-Konfiguration von mongod unter \textit{sudo nano /etc/mongod.conf} auf \textit{authorization: enabled} umgestellt werden. Da die MongoDB von der Gruppe 2 (Kommunikationsmodell) verwendet wird, sollen dafür entsprechend die Firewall-Regeln angepasst werden. Der nachfolgende Auszug aus der Konfiguration-Datei (/root/Firewall.sh) zeigt wie dies beispielhaft erfolgen sollte.
\begin{lstlisting}
# Erlaube zugang im 145.45.x.x port 27017 MongoDB
iptables -A INPUT -s 141.45.0.0/16 -p tcp -m tcp --dport 27017 -j ACCEPT
\end{lstlisting}
Der bereitgestellte Crawler basiert auf Java 11 und ist somit eine plattformunabhängige Lösung, die auf allen Betriebssystemen angeboten werden kann. Die Steuerung des Crawlers erfolgt ausschließlich über die REST-Schnittstelle mit folgenden Abfragen (hier mit curl in der Shell-Konsole):
\begin{lstlisting}
# opendata: https://www.bundestag.de/services/opendata
# Start a default crawl to opendata
$ curl -X POST http://localhost:8080/task/default
# Start the default cron process to opendata: Mon - Fri, 23
$ curl -X POST http://localhost:8080/task/cron
# List all running tasks
$ curl -X GET http://localhost:8080/tasks 
# Cancel planed task
$ curl -X POST http://localhost:8080/task/cancel/{task_id}
# Cancel all planed tasks
$ curl -X POST http://localhost:8080/tasks/cancel
\end{lstlisting}
Der Crawler auf dem infosys1-Server (infosys1.f4.htw-berlin.de) läuft seit dem 27.11.2020, bis auf kleine Unterbrechungen aufgrund von Aktualisierungen, problemlos und konnte zum Zeitpunkt der Verfassung dieser Dokumentation schon über 200 Protokolle herunterladen, parsen und in der MongoDB speichern. Des Weiteren wurden Stammdaten von 4.086 Abgeordneten gesammelt und mehr als 500 URLs durchsucht. Der Datenaustausch mit der Gruppe des Kommunikationsmodells war erfolgreich, da diese ohne Probleme auf die bereitgestellte MongoDB zugreifen konnten und die Benachrichtigungen ebenfalls fehlerfrei empfangen wurden.\\
Aus den vom Crawler gesammelten Daten wird nun durch Gruppe 2 ein Kommunikationsmodell entwickelt, welches von anderen Gruppen für weitere Verarbeitungen und Analysen verwendet werden kann. Der Aufbau dieses Kommunikationsmodells wird nun von Gruppe 2 im nächsten Kapitel aufgegriffen. 

\chapter{Kommunikationsmodell}
\section{Einleitung}\label{sec:08_01_einleitung}
\section{Grundlagen}\label{sec:02_02_grundlagen}
\section{Anforderungsanalyse und Konzept}\label{sec:08_03_anforderungen_konzept}
\section{Implementierung}\label{sec:02_04_implementierung}
\section{Zusammenfassung und Ausblick}\label{sec:07_05_zusammenfassung}

\chapter{Sentiment Analyse}
\section{Einleitung}\label{sec:08_01_einleitung}
\section{Grundlagen}\label{sec:02_02_grundlagen}
\section{Anforderungsanalyse und Konzept}\label{sec:08_03_anforderungen_konzept}
\section{Implementierung}\label{sec:02_04_implementierung}
\section{Zusammenfassung und Ausblick}\label{sec:07_05_zusammenfassung}

\chapter{Analyse der Interaktion zwischen Abgeordneten}
\section{Aufgabe Team 4}\label{sec:04_01_aufgabe}
(Autorin: Jennifer Vormann)\\
\begin{figure}
  \begin{center}
    \includegraphics[width=0.8\textwidth]{Dokumentation-main-2/main_document/chapters/05-Interaktion-Abgeord/sitzverteilung_19wp_bild.jpg}
  \end{center}
  \caption{Sitzverteilung im 19. Deutschen Bundestag \cite{sitze19Bundestag}}
  \label{fig:sitzverteilung}
\end{figure}
\\
\textbf{Team} \\
Rico Stucke | Florian Thom | Jennifer Vormann\\
\\
\textbf{Thema}\\
Graph-basiertes Informationssystem zur Analyse sozialer Interaktion im Deutschen Bundestag\\
\\
\textbf{Aufgabe von Team 4}\\
Unsere Aufgabe bestand darin, die Beziehungen zwischen Abgeordneten darzustellen. Die Beziehungen ausgehend von Abgeordneten zu einer Fraktion und von einer Fraktion zu einem Abgeordneten wurden ebenso einbezogen und betrachtet. Kommentare von einer Fraktion zu einer anderen wurden bei dieser Teilaufgabe außen vor gelassen, da dies Aufgabe von Gruppe 5 war. \\
In der Praxis bedeutete das für uns Daten zwischen unterschiedlichen Datenbanken und Servern zu verarbeiten und zu übertragen. Ziel war es, eine Graphdatenbank zu haben, in der die Interaktionen zwischen den Abgeordneten während der Bundestagssitzungen ersichtlich sind und abgerufen werden können. Es wurden alle Sitzungstage der 19. Wahlperiode betrachtet. Aktuell fanden in dieser Wahlperiode über 200 Sitzungstage statt. Wechsel von Abgeordneten in eine andere Fraktion wurden vorerst außen vor gelassen. Das genaue Vorgehen zur Erreichung dieses Ziels wird im nachfolgenden Kapitel erläutert. Der Hintergrund ist, herauszufinden, ob der Ton im Bundestag tatsächlich rauer geworden ist. Für die Analyse liegen die entsprechenden Sitzungsprotokolle als XML-Dateien vor. Uns wurde von Team 3 Zugang zu einer MongoDB mit den bereits bereits gespeicherten Daten gewährt.
\section{Vorgehen}\label{sec:04_02_vorgehen}
\noindent
(Autorin: Jennifer Vormann)
\subsection{Absprachen}
Im Rahmen der Plenarsitzungen wurde mit allen Gruppen gemeinsam das Vorgehen besprochen. Dies geschah im Rahmen der wöchentlichen Meetings. Zusätzlich fanden diverse Absprachen mit Team 2 und 3 statt. Es musste geklärt werden, wann wir die ersten Daten erhalten und in welcher Form. Mit Team 5 musste abgesprochen werden, ob wir den gleichen Server und/oder die gleiche Datenbank benutzen wollen. Hier wurde sich in beiden Punkten dagegen entschieden. Die Kommunikation mit Team 7 fand ebenfalls ergänzend statt, um die Weiterleitung der Daten zu besprechen.
\subsection{Planung \& Setup}
Zu Beginn des Projektes haben wir ein GitHub Repository erstellt und den virtuellen Server aufgesetzt. Wir haben einen Zeitplan angefertigt und eine Planungspräsentation vorbereitet, die auch anschließend in der Plenarsitzung gehalten wurde. Im Team haben wir uns gemeinsam auf ein Datenbankschema festgelegt. Das Datenbankschema wird im Unterkapitel zum Entwurf näher erläutert.\\
\\
Nachfolgend verständigten wir uns im Team und auch mit Gruppe 5 zu den Technologien und nahmen die entsprechenden Installationen vor. Wir entschieden uns für die Graphdatenbank Neo4j, um die Beziehungen zwischen den Abgeordneten gut visualisieren zu können. Eine weitere Intention war die Lust Neo4j und Cypher neu zu erlernen. Beim Skript entschieden wir uns für ein Python Skript. Das Skript, welches zu implementieren war, sollte eine Verbindung zu den Datenbanken (MongoDB, Neo4j) aufbauen und die Daten einzeln von der MongoDB einlesen. Anschließend sollte es in der Lage sein, die Daten umzuformen - in Nodes, Properties \& Relationships. Als letzter Schritt sollte die Sicherung in die Graphdatenbank Neo4j realisiert werden, um die Daten Gruppe 7 zur Verfügung stellen zu können. Zum Ende unseres Projektes haben die gesamte Umsetzung inklusive Ergebnissen und Ausblick in dieser Dokumentation zusammengetragen. Die Abschlusspräsentation wurde erstellt und vor dem Kurs gehalten.
\section{Entwurf}
(Autor: Florian Thom)
\newline
\newline
Mit diesem Unterkapitel werden Hintergrundwissen, Überlegungen und Prozesse zur Umwandlung von unstrukturierten Daten aus einer NoSQL-Daten\-bank in ein graphenoptimiertes Schema zur Darstellung gegebener Interaktionen zwischen Abgeordneten des Bundestages präsentiert.
\subsection{Überblick}
Diese Arbeit setzt ein Basisverständnis von Graphdatenbanksystemen voraus. Dementsprechend werden zu besagten Graphdatenbanken keine Begriffsbestimmungen oder weitere Erläuterungen vermerkt.
\newline
Ein Graphdatenbankschema ist abhängig von verschiedenen Faktoren. Innerhalb dieser Arbeit wird sich dazu entschlossen auf einen \enquote{Labeled Property Graph} (LPG), statt auf einen Graphen nach dem \enquote{Resource Description Framework} (RDF) zu setzen. Ein Hauptgrund dafür ist, dass man hier versucht sich innerhalb eines PoC zu orientieren.
\newline
Bezüglich des LPGs wird sich für eine Graphdatenbank des Unternehmens \enquote{Neo Technology} mit dem Namen Neo4j entschieden. Innerhalb der LPGs ist Neo4j eine der meist verwendeten Datenbanken. Somit scheint die Verwendung, mit dem angebotenen Support durch Schnittstellen zu verschiedenen Sprachen, angemessen.
\subsection{Graphdatenbank: Neo4j}
Einleitend ist festzustellen, dass Neo4j einen \enquote{native graph storage} umsetzt und eine sogenannte \enquote{index-free adjacency} Datenbank ist \cite{robinson2015graph}. Das bedeutet einerseits, dass der Datenspeicher speziell für Graphen angepasst ist. Neo4j verwendet dafür mehrere sogenannte \enquote{stores} \cite{neostores}. Beispielsweise existiert ein \enquote{store} individuell für alle Knoten oder auch für alle Relationships. Durch den Term \enquote{index-free adjacency} wird nun andererseits beschrieben, dass innerhalb dieser stores direkte physische Pointer, beispielsweise zwischen den Knoten, existieren und diese somit aufeinander zeigen. Im Gegensatz dazu zeigen relationale Datenbanken logisch direkt oder indirekt über IDs aufeinander. Etwaige Optimierungen des Graphen schlagen sich somit direkt auf die Performance des Systems aus.
\newline
Das konkrete Speichern ist mit der Darstellung \ref{fig:Dokumentation-main-2/main_document/chapters/05-Interaktion-Abgeord/image21.png} veranschaulicht. Zu sehen sind ein Node und eine Relationship, sowie eine Andeutung deren interner Speichereinheiten.
\begin{figure}[hbt!]
    \centering
    \includegraphics[width=150px, keepaspectratio]{Dokumentation-main-2/main_document/chapters/05-Interaktion-Abgeord/image21.png}
    \caption{Datenorganisation in Graphdatenbanken - Beispiel Neo4j} \cite{generalStorage}
    \label{fig:Dokumentation-main-2/main_document/chapters/05-Interaktion-Abgeord/image21.png}
\end{figure}
Wichtig ist an dieser Darstellung, dass Nodes ihre Relationships zweifach gruppiert speichern \cite{neo4jrelationshipgrouping}. Einerseits werden die Relationships gruppiert nach ihrem Namen (auch Typ) gespeichert. Andererseits wird innerhalb jeder dieser Gruppen differenziert zwischen eingehenden- und ausgehenden Relationships. Durch diese zweifache Gruppierung sollten unter anderem Performanceeinbußen durch Inselbildung nach Typ der Relation oder nach Richtung der Relation vernachlässigbar sein. So muss beispielsweise nicht durch alle eingehenden Relationships iteriert werden, wenn man lediglich nach Ausgehenden sucht.
\subsection{Graphdatenbankschema: Entwurf}
Der Entwurf eines Graphdatenbankschemas ist von einigen Faktoren, u.a. Technischen und Inhaltlichen, abhängig. Diese werden folgend hervorgehoben und definieren eine Art Ausgangslage.
\newline
Die Datenbasis ist insofern für die aktuelle Aufgabenstellung wichtig, als das ohne Verständnis für die bestehenden Datenfelder, das Modellieren eines neuen Modells mit neuen Datenfeldern schwierig ist. Ein Überblick über das Schema wird in der Ausarbeitung von Gruppe 2 gegeben.
Außerdem ist die These des Projektes zu beachten. Es ist somit darauf zu achten, dass gerade Fragestellungen, die eben in Richtung dieser These gehen, angemessen gut abfragbar sein müssen.
\newline
Abschließend werden die konkreten Fragestellungen, die an die Graphdatenbank gestellt werden sollen mit den entsprechenden Verantwortlichen ermittelt. Zum Zeitpunkt der Ermittlung, soll beispielsweise zu einigen Eigenschaften ein \enquote{Score} berechnet werden, wie zum Sender, die viele Nachrichten mit positivem Sentiment verschicken \cite{initialQuestionsGroup7}. Diese \enquote{Scores} sollen (teilweise) für jeweilige Zeitabschnitte berechnet werden, beispielsweise pro Monat. Dies ist insofern relevant, als dass gerade zeitliche Abschnitte vermutlich eine entscheidende Rolle einnehmen.
\newline
Ein erster, eher umfangreicher, Entwurf ist in Abbildung \ref{fig:Dokumentation-main-2/main_document/chapters/05-Interaktion-Abgeord/image8.png} zu sehen. Dieser und folgende Entwürfe wurden nach eigenem Ermessen strukturiert, wobei sich teilweise an Roy-Hubara et al. \cite{graphdatabasemodelling} orientiert wurde.
\begin{figure}[hbt!]
    \centering
    \includegraphics[width=375px, keepaspectratio]{Dokumentation-main-2/main_document/chapters/05-Interaktion-Abgeord/image8.png}
    \caption{Initialer Entwurf Graphdatenbankschema}
    \label{fig:Dokumentation-main-2/main_document/chapters/05-Interaktion-Abgeord/image8.png}
\end{figure}
Der Grafik sind besonders drei Kernüberlegungen zu entnehmen. Innerhalb der Hauptstruktur hat eine Person eine oder mehrere Fraktionen und eine oder mehrere Parlamentsrollen. Die zweite Kernüberlegung ist die Einführung eines sogenannten \enquote{Timeline trees} (TLT) \cite{robinson2015graph} der mit der Sitzung und den jeweiligen Kommentaren verbunden ist. Er verfügt über die Stufen \enquote{:Year}, \enquote{:Month} und \enquote{:Day}. Grund für die Integration ist das Wissen über die Wichtigkeit des Faktors Zeit für die Folgeanalysen. Ohne besagten TLT wäre eine Laufzeit von O(N) zu erreichen. Mit Wissen zu TLT könnte man diese Struktur ausnutzen, um die Zugriffszeit für den Erhalt einer gefilterten Menge deutlich zu erhöhen. Beispielsweise könnte man den Zugriff auf alle Kommentare des Monats Februar im Jahr 2020 von O(N) auf in etwa O(2) durch den Zugriff 2020:Year $\rightarrow$ Februar:Month optimieren.
\newline
Eine dritte Kernüberlegung besteht darin, das Sentiment als externen Knoten aus dem Knoten Kommentar zu extrahieren. Wenn man das eher stetige Sentiment (z.B. 0,34216754) in ein eher diskretes Sentiment (z.B. 0,34) transformiert, ist ein Vorteil bei dem Filtern nach einem bestimmten Sentiment erkennbar (Das Sentiment bewegt sich im Bereich [-1.0, 1.0]). Da diese Transformation bei der Umwandlung der Daten aus der NoSQL-Datenbank in die Neo4j sehr gut möglich ist, sollte diese Überlegung nicht unbemerkt gelassen werden. Da hier beispielsweise nun auf zwei Dezimalstellen nach dem Komma gerundet wurde und das Sentiment sich im Bereich [-1.0, 1.0] bewegt, existieren so 200 verschiedene Gruppen, beziehungsweise demzufolge 200 verschiedene Knotentypen. Der Vorteil wird an einem Beispiel demonstriert. Möchte man eine Menge von Kommentaren erhalten, deren Sentiment positiv ist, müsste man bisher über alle Kommentare iterieren (ca. 70000-250000). Mit der soeben vorgestellten Methode würden sich diese Iterationen auf 100 reduzieren (alle 100 positiven Sentiment-Knoten z.B. 0.01, 0.02, 0.03, ...). Der Basisentwurf ist in Abbildung \ref{fig:Dokumentation-main-2/main_document/chapters/05-Interaktion-Abgeord/image9.png} dargestellt.
\begin{figure}[htb]
    \centering
    \includegraphics[width=250px]{Dokumentation-main-2/main_document/chapters/05-Interaktion-Abgeord/diagramm.png} 
    \caption{Datenbankschema}
    \label{fig:Dokumentation-main-2/main_document/chapters/05-Interaktion-Abgeord/image9.png}
\end{figure}
Innerhalb der Hauptstruktur ist erkennbar, dass nun zwischen Knoten mit dem Label \enquote{:Commentary} und \enquote{:Faction} eine eingehende- und eine ausgehende Relationship hinzugefügt wurde. Der Grund dafür ist, dass nun teilweise Interaktionen zwischen einem Abgeordneten (einer Person) und einer Partei als Gesamtes identifiziert wurden. Darüber hinaus wurde auf den TLT verzichtet. Grund dafür war, dass dieser anscheinend schwieriger umzusetzen ist.
\section{Umsetzung}\label{sec:04_03_umsetzung}
(Autor: Rico Stucke)
\subsection{Lesen aus MongoDB}
Zunächst haben wir uns mit dem Empfang der Daten aus der MongoDB von Team 3 beschäftigt. Für die Abfrage der Daten aus der MongoDB benutzen wir die pymongo Library von Python. Der Client öffnet eine Verbindung zur Datenbank von Gruppe 3 und liest alle Collections, die sich in der Datenbank befinden. Jede Collection entspricht dabei einem Sitzungstag. Im nächsten Schritt aggregieren wir alle ausgelesenen Werte, formen diese um und entfernen eventuelle Dopplungen, die vor allem bei wiederkehrenden Personen oder Fraktionen auftreten. Am Ende dieses Prozesses entstehen Dictionaries, die alle einzigartigen Personen und Fraktionen, sowie Sitzungstage und Kommentare, beinhalten.
\subsection{Nodes \& Relations}
Mit der neomodel Library, die als Object Graph Mapper dient, können Klassen definiert werden, die dann von neomodel als Knoten in Neo4j übernommen werden. Außerdem können an diesen Klassen auch die Beziehungen zwischen den Klassen definiert werden. Die entstandenen Dictionaries dienen als Datengrundlage für die Knotenklassen in neomodel.
\begin{figure}[htb]
    \centering
    \includegraphics[width=10cm]{Dokumentation-main-2/main_document/chapters/05-Interaktion-Abgeord/faction.jpg} 
    \caption{Beispiel einer neomodel Klasse}
    \label{fig:commentary}
\end{figure}
Abbildung 5 zeigt beispielhaft eine Klasse aus dem Programm, um zu verdeutlichen wie neomodel verwendet wird, um einen Node in der Datenbank zu definieren. Durch die Implementierung von StructuredNode wird neomodel mitgeteilt, dass eine Klasse ein Node in der Datenbank sein soll und über die RelationshipTo und RelationshipFrom Funktionen können die Relationen definiert werden, die von diesem Knoten ausgehen oder eingehen.
\subsection{Neo4j DB Connection}
Die Python library neomodel bietet für das Erzeugen von Nodes und Relations in Neo4j ein einfaches Interface, dass durch seine StructuredNode-Klasse implementiert wird. Dadurch wird es möglich einen Node, der durch eine Klasse, die StructuredNode implementiert, dargestellt wird, über zwei Funktionen in der Datenbank zu speichern und Relationen mit anderen Node-Klassen aufzubauen.\\
Das Speichern Erfolg durch die von StructuredNode geerbte Methode save(). Hierbei sind keine weiteren Angaben nötig. Diese Methode muss nur vom jeweilig instanziierten Objekt aufgerufen werden, um es in die Datenbank zu übertragen. Für die Herstellung einer Relation muss die connect()-Methode aufgerufen werden. Wichtig hierbei ist, dass die Klasse ein Attribut mit dem Namen der Relation implementieren muss. Der Funktionsaufruf erfolgt dann in dem folgenden Schema:\\
\newcommand\tab[1][1cm]{\hspace*{#1}}
\tab ObjektA.relation\_name.connect(ObjektB)
\subsection{Laufzeitverbesserungen}
In seiner ersten Iteration hat das Programm für die Übertragung einer Collection aus MongoDB in die Neo4jDB rund 5 Minuten gebraucht. Da eine Collection aber nur einen Sitzungstag umfasst, wurde schnell klar, dass Verbesserungen an der Laufzeit nötig sind. Durch Änderungen am Code konnte die Geschwindigkeit gesteigert werden indem nicht mehr für jede Datenbankaktion eine Transaktion verwendet wird, sondern mehrere Aktionen gebündelt werden in einer Transaktion.\\
Bei einem Test mit einer lokalen Neo4j Datenbank, die außerhalb des HTW-Netzes lief, konnte auch beobachtet werden, dass das Programm wesentlich schneller läuft. Hierbei handelte es sich um Differenzen von 8 Schreibaktionen in der Datenbank pro Sekunde, wenn das Programm von einem privaten Rechner gestartet wurde, der über einen Tunnel mit dem HTW-Netz verbunden war, zu über 200 Schreibaktionen wenn das Programm auf dem selben Rechner lief, der auch die Datenbank gehostet hat. Diese Beobachtung führte dann dazu, dass das Programm um eine einfache REST-API und einen simplen Flask-Server erweitert wurde. Dies ermöglichte es das Programm direkt auf der Hardware laufen zu lassen, die auch die Neo4j Datenbank enthält im HTW-Netz. Durch diese Änderungen gelang es uns die Laufzeit von ursprünglich geschätzten 13 Stunden auf rund 25 Minuten zu reduzieren.
\section{Ergebnisse}\label{sec:04_04_ergebnisse}
(Autoren: Jennifer Vormann | Rico Stucke)\\
\\
In seiner jetzigen Form erzeugt das Skript die Neo4j Datenbank in 25 Minuten. Dabei werden rund 278000 Knoten und 830000 Beziehungen zwischen diesen Knoten angelegt. Den größten Anteil dabei haben mit rund 276000 Knoten die Kommentare der Abgeordneten und Fraktionen. Die lange Laufzeit des Programms lässt sich erklären durch das Fehlen eines Batch-Prozesses für die Inserts in die Datenbank. Neomodel bietet zwar eine Möglichkeit für Batch-Inserts, aber es besteht dabei keine Möglichkeit gleichzeitig die Beziehungen mit anzulegen.\\
\begin{figure}[htb]
    \centering
    \includegraphics[width=8cm]{Dokumentation-main-2/main_document/chapters/05-Interaktion-Abgeord/5.PNG} 
    \caption{Ausgabe eines Kommentars}
    \label{fig:commentary}
\end{figure}
Ein weiteres Ergebnis und deren Visualisierung zeigt die Abbildung der Personen und des Sendens eines Kommentars in Richtung einer anderen Person oder Fraktion. Wir haben uns für die Darstellung dieses Graphen in die Dokumentation entschieden, weil der Fokus der Arbeit auf den Kommentare und deren Sentiments liegt. Die Graphen, die in Neo4j angezeigt werden können, sind schnell sehr komplex und können nur in kleinen Mengen (default: 300 Nodes) dargestellt werden.
\begin{figure}[htb]
    \centering
    \includegraphics[width=\textwidth]{Dokumentation-main-2/main_document/chapters/05-Interaktion-Abgeord/6.PNG} 
    \caption{Neo4J Graph -> Person - Sender - Commentary - Receiver - Person}
    \label{fig:personcomment}
\end{figure}
\section{Ausblick}\label{sec:04_05_ausblick}
(Autoren: Jennifer Vormann | Florian Thom)\\
\\
Künftig wäre es eines unserer Ziele, die Laufzeit der Anwendung zu verbessern. Hierzu könnte generell noch einmal über den Code gegangen und dieser optimiert werden. So könnte die Runtime gegebenenfalls verbessert werden, um schnelles Schreiben in die Graphdatenbank zu ermöglichen. 
Zusammenfassend können folgende Punkte für eine Weiterentwicklung analysiert und gegebenenfalls umgesetzt werden:\\
- Arbeiten mit Batch\\
- Libraries | Programmiersprache variieren \\
\\
Ebenso wäre es ein Anliegen, die weiteren Wahlperioden zu ergänzen. Bei Optimierungen des Codes würden wir aller Wahrscheinlichkeit nach auf die Verwendung von neomodel verzichten und auf die Standard Implementierung mit Neo4j Python Driver ausweichen. Bei der Erweiterung um andere Wahlperioden wäre zu überdenken, die Wahlperiode als eigenen Node zu erstellen. Dazu müsste das Datenbankschema entsprechend angepasst werden.\\
Gerade im Entwurf wurden gewisse Optimierungen vorgestellt, bei denen aus theoretischer Sicht teilweise gute Verbesserungsmöglichkeiten gesehen wurden. Anfänglich wurde über einen praktischen Test oder Vergleich der Schemata spekuliert. Dieser Test blieb leider aufgrund von gewissen Schwierigkeiten in der Implementierung aus und könnte perspektivisch noch durchgeführt werden. Verbesserungsmöglichkeiten sind aus unserer Perspektive bezüglich des finalen Entwurfes vorhanden. Diese bestehen besonders in der Einbindung eines TLTs. Dies wurde mit dem ersten Entwurf aus theoretischer Sicht erläutert.
\section{Lernziele}\label{sec:04_07_lernziele}
(Autorin: Jennifer Vormann)\\
\\
Die Lernziele dieses Projektes und des damit verbundenen Kurses waren sehr vielfältig. In den Vorlesungen wurde Grundlagenwissen in den folgenden Bereichen vermittelt:\\
- Informationssysteme allgemein \& Dateninspektion\\
- Crawling Textanalyse \& NoSQL\\
- Sentiment Analyse \& Graphdatenbanken\\
\\
Aufgrund der Art der Durchführung der Übungen konnte das Format der Plenarsitzungen, inklusive deren Rollen, erlernt werden. Das Einsetzen realer Daten und deren Verarbeitung gehört zu den erreichten Lernzielen, ebenso wie das Extrahieren der relevanten Informationen aus den gegebenen Daten. Die Daten mussten betrachtet und bewertet werden anhand ihrer Relevanz für das Ziel der gesamten Projektaufgabe. Anschließend wurde das Aggregieren, Verknüpfen und Darstellen der Daten erlernt.\\
Zur Erstellung des Datenbankschemas wurden Kenntnisse zu Klassendiagrammen in Notation der UML vertieft und angewandt. Für die Zwischen- und Abschlusspräsentation der Vorgehensweise, konnte unser Wissen über das Schreiben eines Proof of Concept und das ansprechende Erstellen von Präsentationen gefestigt werden. Wir haben den Umgang mit Neo4j und deren deklarative Abfragesprache Cypher neu erlernt. Bei der Verarbeitung der Daten die von Gruppe 3 geliefert wurden und beim Weiterleiten unserer Ergebnisse an Gruppe 7 haben wir erfolgreich Schnittstellen von verschiedenen Datenbanken hergestellt. Unser Vorgehen und die Wahl der Technologien für die gesamte Umsetzung unserer Teilaufgabe lagen komplett in unserer Hand. Daher mussten wir abwägen, recherchieren und Vor- und Nachteile evaluieren. Die Kommunikation und Kollaboration war nicht nur in unserem Team wichtig sondern auch teamübergreifend. Auch das Verfassen einer kursumfassenden Dokumentation benötigte ein merklich höheres Maß an Abstimmung, Organisation und Kommunikation. Eigenständiges Arbeiten und Praktiken des Zeitmanagements konnten festigen werden.


\chapter{Analyse der Interaktion zwischen Fraktionen}
\section{Einleitung}\label{sec:08_01_einleitung}
\section{Grundlagen}\label{sec:02_02_grundlagen}
\section{Anforderungsanalyse und Konzept}\label{sec:08_03_anforderungen_konzept}
\section{Implementierung}\label{sec:02_04_implementierung}
\section{Zusammenfassung und Ausblick}\label{sec:07_05_zusammenfassung}

\chapter{Graphauswertung}
\normalsize \textsc{Ermittlung der Stimmungsmacher im Bundestag} \\
\normalsize \textsc{Markus Christopher Glutting, Marie Bittiehn, Miriam Lischke}
\section{Einleitung}\label{sec:08_01_einleitung}
\section{Grundlagen}\label{sec:02_02_grundlagen}
\section{Anforderungsanalyse und Konzept}\label{sec:08_03_anforderungen_konzept}
\section{Implementierung}\label{sec:02_04_implementierung}
\section{Zusammenfassung und Ausblick}\label{sec:07_05_zusammenfassung}

\chapter{Benutzeroberfläche}
Die Gruppe 8 hat sich mit der Erstellung der Benutzeroberfläche beschäftigt. Das Team bestand aus den vier Studenten, Amanda Joelle Dzukou Kom, Steven Mi, Tilman Möller und Oliver Kütemeier. Die Implementation des Projektes wurde in enger Zusammenarbeit der Studenten umgesetzt, um eine gleichmäßige Verteilung der Kompetenzen zu sichern. 
\section{Einleitung}\label{sec:08_01_einleitung}
\section{Auswahl des Frameworks}%\label{sec:08_02_Auswahl-des-Frameworks}
Bei der Implementation des Frontends bestand der erste Schritt in der Auswahl des Frameworks. Die drei betrachteten Kandidaten waren Vue.js, Angular und React.

\begin{figure}[hbt!]%
  \centering
  \subfigure{
  \includegraphics[width=3cm]{images/08-Benutzeroberfläche/08-Angular_logo.png}
  }%
  \qquad
  \subfigure{
  \includegraphics[width=3cm]{images/08-Benutzeroberfläche/08-Vue_logo.png}
  }%
  \qquad
  \subfigure{\includegraphics[width=4cm]{images/08-Benutzeroberfläche/08-react_logo.png}}%
  \caption{Logos der zur Auswahl stehenden Frameworks}%
\end{figure}

Nach erster Recherche wurde klar, dass alle Frameworks in der Lage sind moderne und stabile Webseiten zu erstellen. Sie unterstützen alle diverse Bibliotheken zur Visualisierung und bieten die Funktionalitäten die geplanten Inhalte umzusetzen. Da die drei Frameworks die technischen Anforderungen gleichmäßig erfüllen, konnten wir persönliche Kriterien aufstellen um eine Entscheidung zu treffen. 
Das erste Kriterium bestand in der aktuellen Nachfrage in der Wirtschaft. Dazu haben wir die Menge der Suchanfragen nach dem jeweiligen Skill in unterschiedlichen Bereichen untersucht. Angefangen haben wir mit den Jobbörsen und Business Social Networks.

\begin{figure}[hbt!]%
  \centering
  \includegraphics[width=14cm]{images/08-Benutzeroberfläche/08-jobbörsen-anfragen.PNG}
  \caption{Vergleich der Suchanfragen in Jobbörsen und Business Social Networks}%
\end{figure}

Hier ist klar, dass die Industrie mehr an den Frameworks React und Angular interessiert ist. Um die allgemeine Meinung besser einschätzen zu können haben wir die Google Trending Suchanfragen der verglichen. 

\begin{figure}[hbt!]%
  \centering
  \includegraphics[width=14cm]{images/08-Benutzeroberfläche/08-google_trending.PNG}
  \caption{Vergleich der Google Trending Suchwörter}%
\end{figure}

Auch hier stehen Angular und React weit vor den Anfragen von, sodass wir uns zwischen diesen beiden entscheiden mussten. Als letztes Kriterium haben wir die aktuellen Github Statistiken der Frameworks verglichen um einen Eindruck für die Community und dessen Aktivität zu erhalten.

\begin{figure}[hbt!]%
\centering
  \includegraphics[width=14cm]{images/08-Benutzeroberfläche/08-github_stats.PNG}
  \caption{Vergleich der Github Statistiken der Frameworks}%
\end{figure}

Unter Berücksichtigung aller Kriterien haben wir uns für die Entwicklung der Website mit React entschieden.
\section{Infrastruktur}\label{sec:08_03_Infrastruktur}
Die Infrastruktur, welche unser Team zum verwalten des Quellcodes und zum bereitstellen der Webseite genutzt hat, war zum einen an die Absprachen der Studentengruppe und zum anderen an die Vorgaben der Hochschule gebunden. In Abstimmung mit allen Studenten wurde sich dazu entschieden, den Quellcode in Github zu verwalten. Die Bereitstellung erfolgte darüber hinaus über die Server der HTW Berlin, auf denen wir eine Virtuelle Maschine zur Verfügung gestellt bekommen haben.

\subsection{Quellcode Verwaltung}
Innerhalb der Github Organisation "Sentiments-of-Bundestag" haben wir das Git Repository "frontend\_sentiment" angelegt. Wir haben mithilfe der von Github eingeführten Issues eine Schnittstelle zum Kommunizieren von Problemen angelegt, die von den anderen Gruppen genutzt werden konnte. Über die Readme des Projekts haben wir alle wichtigen Befehle und Hilfestellungen zur Nutzung der Anwendung dokumentiert.
Durch integrieren eines Linters in den Build-Prozess der Anwendung konnten wir als Gruppe feste Quellcode-Formatierungs-Regeln definieren und befolgen um Clean Code zu implementieren. Als zusätzliche Regelung zur Unterbindung von Problemen beim Programmieren haben wir in Branches gearbeitet und bei Änderungen am Quellcode Rücksprache im Team gehalten. 

\subsection{Bereitstellung}
Bei der Bereitstellung der Anwendung haben wir die Serverkapazitäten der HTW Berlin genutzt. Uns wurde eine virtuelle Maschine zur Verfügung gestellt. Auf dieser mussten zunächst die Zugriffsregeln der Firewall angepasst werden um den Zugriff auf feste Ports, von außerhalb des HTW Netzwerks, zuzulassen. Außerdem mussten Git und Node.js installiert werden. Git wurde benötigt um den Quellcode aus dem Repository bei Änderungen herunterzuladen. Node.js wurde als Server zur Bereitstellung der React Anwendung benutzt.
\section{Design Thinking}\label{sec:08_04_Design_Thinking}
Beim Gestalten der Benutzeroberfläche haben wir uns an den sechs Phasen des von Tim Browns erfundenen Design Thinking Prozess orientiert. 

\begin{figure}[hbt!]
    \centering
    \includegraphics[width=14cm]{images/08-Benutzeroberfläche/08-Design Thinking.png}
    \caption{Ablauf des Design Thinking Prozess}
\end{figure}

Dieser Prozess wird im Online Artikel "Was ist Design Thinking?"\cite{DeTh} gut beschrieben. 
In diesem Projekt standen wir zunächst vor dem Problem, dass wir die Daten und fest definierten Schnittstellen erst im letzten drittel des Semesters vorliegen hatten. Vorher haben wir uns mit der Auswahl der Graphen- und Diagrammtypen beschäftigt. Wir haben Prototypen entwickelt und diese in regelmäßigen Abständen mit der Studentengruppe geteilt und Rückmeldungen gesammelt, welch in den Design Prozess eingeflossen sind. Mit den Rückmeldungen der Gruppe konnten wir unseren Prototypen verbessern, Teile umgestalten Beschreibungen anpassen oder Bereiche entfernen.
Die erste Iteration bestand in der Vorbereitung der Planungspräsentation unserer Gruppe. Bis zu diesem Zeitpunkt haben wir Ideen gesammelt, wie die angekündigten Daten in Diagrammen und für Benutzer*innen verständlich aufbereitet werden können. In einem Prototypen haben wir vier unterschiedliche Diagramme vorbereitet.

\begin{figure}[hbt!]%
    \centering
    \begin{subfigure}
        \centering
        \includegraphics[width=5cm]{images/08-Benutzeroberfläche/08-flowchart.PNG}
    \end{subfigure}
    \qquad
    \begin{subfigure}
         \centering
         \includegraphics[width=5cm]{images/08-Benutzeroberfläche/08-Graphen.PNG}
    \end{subfigure}
    \qquad
    \begin{subfigure}
         \centering
        \includegraphics[width=5cm]{images/08-Benutzeroberfläche/08-Zusammensetzung_bundestag.PNG}
  \end{subfigure}
  \qquad
  \begin{subfigure}
        \centering
        \includegraphics[width=5cm]{images/08-Benutzeroberfläche/08-Sehnendiagramm.PNG}
  \end{subfigure}
  \caption{Vorbereitete Diagramme der ersten Iteration}%
\end{figure}

Die Rückmeldungen und Diskussion am Ende der ersten Iteration, haben gezeigt, dass die direkte Visualisierung des Graphen zu unübersichtlich und nicht anschaulich ist. Ebenso ist das Sankey-Diagramm unübersichtlich und schwer zu interpretieren. Diese Diagramme sind nicht übernommen worden. Im Gegenteil dazu, wurden das Sehenendiagramm zur Visualisierung der Interaktionen und das halbe Kreisdiagramm zur Darstellung der Sitzeverteilung sehr positiv angenommen und deswegen weiter entwickelt.

In der zweiten Iteration wurden das Netzdiagramm und die Balkendiagramme integriert, die positives Feedback bekommen haben. Ebenfalls haben wir mit Box-Plots experimentiert, diese allerdings wegen der hohen Komplexität nicht weiter verfolgt.

\begin{figure}[hbt!]
    \centering
    \includegraphics[width=8cm]{images/08-Benutzeroberfläche/08-Netzdiagramm.PNG}
    \caption{Eingeführtes Netzdiagramm}
\end{figure}

Nach den ersten beiden Iterationen standen die Diagrammtypen fest und wir konnten mit den einbetten der Diagramme in die Oberfläche beginnen. Dazu wurden die Diagramme in ein Storyboard eingebunden und mit Texten beschrieben. In den letzten Iterationen wurden Kleinigkeiten, wie die Texte und die Farben, abgestimmt um zum finalen Produkt zu gelangen.
\section{Implementierung}\label{sec:08_05_implementierung}

\subsection{Technologien}
Die Benutzeroberfläche ist eine Single-Page-App, die hauptsächlich in React mit TypeScript geschrieben wurde. TypeScript typisiert JavaScript und ermöglicht die Vorteile einer typisierten Sprache. React Router DOM wurde verwendet, um die Navigation ohne eine Webseitenaktualisierung zu implementieren. Datenfluss und -verwaltung wurden mit Redux und Redux Sagas ermöglicht.  Durch den Einsatz der vordefinierten UI-Elemente von Ant Design und Bootstrap konnten wir eine einheitliche Designsprache beibehalten. Letztlich wurden mit nivo die dynamischen und animierten Grafiken erstellt.


\begin{figure}[hbt!]%
  \centering
  \includegraphics[width=14cm]{images/08-Benutzeroberfläche/08-frontend-framework-icons.PNG}
  \caption{Eine Übersicht über alle verwendeten Technologien, Werkzeuge und Bibliotheken}%
\end{figure}



\subsubsection{React}
React basiert auf dem sogenannten Komponentenmuster (engl. \dq component pattern\dq), in der jedes Element einer Webseite einer \dq stateless\dq Komponente entspricht. \dq Stateless\dq Komponenten sind wiederverwendbare Webseiten Elemente, welche keinen eigenen Status, Daten oder Inhalt besitzen. Sie dienen lediglich für die funktionale Transformationen von Eingabedaten.\cite{ReactDoc} Zudem kann jede Komponente ebenfalls eine Tochterkomponente besitzen.

Die Daten und Komponenten werden von sogenannten Container verwaltet. Ein Container aggregiert die benötigten Komponenten und sorgt für den Datenfluss.

\begin{figure}[hbt!]%
  \centering
  \includegraphics[width=10cm]{images/08-Benutzeroberfläche/08-react-component.PNG}
  \caption{Die Visualisierung vom Datenfluss im Komponentenmuster. Eine Menge von Container besitzen Komponenten welche Daten erhalten und weiter an den Tochterkomponenten übergeben von \cite{itnext18}}%
\end{figure}

\subsubsection{Redux und Redux Sagas}
Redux und Redux-Sagas wurde für den Datenfluss und -verwaltung verwendet. 

Redux dient als ein globaler Speicherort für Informationen. Daher wird Redux oftmals auch als Redux Store bezeichnet. Container greifen auf den Redux Store zum um entweder Daten zu erhalten oder zu verändern. Dabei können Container sogenannte Actions ausführen, um Daten zu erhalten. Mithilfe von Reducer können Daten vorselektiert oder verändert werden. 

\begin{figure}[hbt!]%
  \centering
\includegraphics[width=10cm]{images/08-Benutzeroberfläche/08-react-redux.png}
  \caption{Die Visualisierung vom Datenfluss im Komponentenmuster mit Redux von \cite{itnext18}}%
\end{figure}

Die Informationen im Redux Store müssen jedoch erst von einem Server angefordert werden. Dieser Vorgang ist jedoch asynchron und gibt keine Informationen über den Status des Prozesses zurück. Beispielsweise es ist nicht bekannt ob der Server die Daten verarbeitet oder ob der Prozess fehlgeschlagen ist. Der Anwender hingegen möchte wissen, ob Daten gerade verarbeitet werden. Um dies zu ermöglichen, wurde Redux-Sagas eingesetzt. Redux-Sagas ist eine Middleware, die sich zwischen der Benutzeroberfläche und dem Redux Store befindet. Wenn die Benutzerfläche eine Action ausführt, wird die Middleware Daten vom Server anfordern und dem Reducer weitergeben. Der Reducer wiederum bearbeitet die Daten und speichert sie im Redux Store. 

\begin{figure}[hbt!]%
  \centering
\includegraphics[width=10cm]{images/08-Benutzeroberfläche/08-redux-middleware-diagram.png}
  \caption{Die Visualisierung vom Datenfluss mit Redux und Redux-Sagas von \cite{scalac19}. Die View ist die Benutzeroberfläche, welche eine Action ausführt und den Prozess startet.}%
\end{figure}



\subsubsection{Ant Design, Bootstrap und nivo}
Ant Design, Bootstrap und nivo sind sogenannte UI-Komponenten-Bibliotheken, Bibliotheken die vorgefertigte UI-Komponenten wie Buttons, Eingaben, Dialoge etc. bereitstellen. Sie dienen als Bausteine für Layouts und können modular verwendet werden. Jede Bibliothek hat ihre eigene Designsprache und damit ein charakteristisches Aussehen. Sie können jedoch in der Regel bis zu einem gewissen Grad angepasst werden.

Für Sentiment of Bundestag wurden Ant Design, Bootstrap und nivo verwendet. Ant Design wurde von der Firma Ant Design entworfen und stellt allgemeine UI-Komponenten wie Buttons, Tabellen und co. in der firmeninternen Designsprache zur Verfügung. Bootstrap stellt ähnliche Komponenten wie Ant Design zur Verfügung, allerdings in der Designsprache der Firma Twitter. Bei Nivo handelt es sich um ein Open-Source-Projekt, das vorgefertigte Diagramme anbietet, die mit d3.js und anderen React-Frameworks implementiert sind. Ihr Vorteil ist, dass die Diagramme animiert und für mobile Geräte optimiert sind. 

\section{Zusammenfassung und Ausblick}\label{sec:08_05_zusammenfassung}
Durch die unterschiedlichen angewandten Methoden und die verwendeten Frameworks und Bibliotheken konnten wir viel Neues kennen lernen. Rückblickend konnten wir in diesem Semester eine solide und gut vorzeigbare Benutzeroberfläche für unser Informationssystem schaffen. Wir haben sehr viel positives Feedback erhalten und bei der Live-Demo und in den vorangegangenen Präsentationen hat die Webseite wie geplant funktioniert.

\begin{figure}[hbt!]
    \centering
    \includegraphics[width=14cm]{images/08-Benutzeroberfläche/08-Legislaturperioden.PNG}
    \caption{Screenshot aus der Benutzeroberfläche}
\end{figure}

Ein Aspekt den wir zu Beginn des Semesters unterschätzt haben war die Abhängigkeit von den vorangehenden Gruppen. Diese führte zu Zeitproblemen, Koordinationsaufwand und Problemen bei der Entwicklung. Diese Herausforderungen konnten wir nur im Team durch Zusammenarbeit und durch direkte Kommunikation mit den anderen Gruppen lösen.
Alles in Allem sind wir mit dem Projekt sehr Zufrieden. Sowohl der Lernerfolg in diesem Semester als auch das dabei entstandenen Produkt sind überzeugend. Wir hoffen sehr, dass das Projekt in Zukunft fortgeführt wird und andere Studierende den Design Thinking Prozess und die Entwicklung fortführen.

\subsection{Erreichbarkeit}
Die Webseite kann, für derzeit unbestimmte Zeit, unter http://infosys7.f4.htw-berlin.de/ erreicht werden. Sollte dieser Link nicht mehr erreichbar sein, bitte wenden Sie sich an Prof. Dr. rer. nat. Thomas Hoppe.




\pagenumbering{roman}
\setcounter{page}{6}

\printbibliography[title={Literaturverzeichnis}]
\addcontentsline{toc}{chapter}{Literaturverzeichnis}
\newpage

\printglossaries
\addcontentsline{toc}{chapter}{Glossar}
\newpage

\end{document}
