\section{Einleitung}\label{sec:06_01_einleitung}
Eine Fraktion bezieht sich auf eine freiwillige Vereinigung von mindestens fünf Abgeordneten des Bundestages, die ohne Konkurrenz
ihre politische Ziele gemeinsam verfolgen und i.d.R derselben Partei angehören.
Es ist darauf hinzuweisen, dass Fraktionen nicht mit Parteien verwechselt werden sollen. Die Erstere dient dem öffentlichen Interesse und die Letztere dem Privatinteresse.
Fraktionen spielen eine entscheidende Rolle im Deutschen Bundestag. 
Sie dürfen u.a. Gesetzentwürfe bzw. Änderungsanträge von solchen einbringen als auch kleine und große Anfragen im Bundestag stellen. 
Das bedeutet, dass Fraktionen und deren gegenseitigen Aussprachen ``die politische Willensbildung maßgeblich mitbestimmen''.\newline 
Die Zusammenfassung der Stimmungen aller Abgeordneten in derselben Fraktion, und zwar für jede Fraktion, ermöglicht
die Erstellung eines Graphen, dessen Knoten und Kanten jeweils den Fraktionen und deren Beziehungen entsprechen. 
Der Graph kann demzufolge analysiert werden, um unterschiedliche Muster in den Interaktionen zwischen zwei Fraktionen zu finden.
Diese Arbeit beschäftigte sich mit der Erstellung einer Applikation, die den vorgenannten Graph in automatisierter Art und Weise aus einem vorverarbeiteten Datenmodell ableitet.\newline 
Da das bereitgestellte Datenmodell unterschiedliche Typen von Beziehungen enthält, war ein mengentheoretisches Verfahren zur Aggregation der Daten notwendig. Das Verfahren ist in Kapitel 2 näher beschrieben. 
Kapitel 3 enthält die funktionale bzw. nicht-funktionale Anforderungen der Applikation und Kapitel 4 beschreibt deren Umsetzung und die dazu genutzten Technologien. 
Kapitel 5 fasst die Ergebnisse zusammen und beschreibt verschiedene Möglichkeiten zur Optimierung der Applikation.  


