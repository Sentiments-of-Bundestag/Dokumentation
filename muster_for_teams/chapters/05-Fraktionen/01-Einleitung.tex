\section{Einleitung}\label{sec:06_01_einleitung}
Eine Fraktion bezieht sich auf eine freiwillige Vereinigung von mindestens fünf bis zur hundert Abgeordneten des Bundestages, die ohne Konkurrenz
ihre politische Ziele gemeinsam verfolgen und i.d.R derselben Partei angehören.
Es ist darauf hinzuweisen, dass Fraktionen nicht mit Parteien verwechselt werden sollen. Die Erstere dient dem öffentlichen Interesse und die Letztere dem Privatinteresse.
Das heißt, dass Fraktionen aus öffentlichen Mitteln finanziert werden.
Außerdem sind Fraktionsmitglieder gesetzmäßig an Aufträge und Weisungen nicht gebunden und nur ihrem Gewissen unterworfen.\newline 
Fraktionen spielen eine entscheidende Rolle im Deutschen Bundestag. 
Sie dürfen u.a. Gesetzentwürfe bzw. Änderungsanträge von solchen einbringen, kleine und große Anfragen im Bundestag stellen, und Sondersitzungen des Bundestags erzwingen. 
Dies deutet darauf hin, dass Fraktionen und deren gegenseitigen Aussprachen ``die politische Willensbildung maßgeblich mitbestimmen''. 
\subsection{Aufgabenstellung}
Die Zusammenfassung der Stimmungen aller Abgeordneten in derselben Fraktion, und zwar für jede Fraktion, ermöglicht
die Erstellung eines Graphen, dessen Knoten und Kanten jeweils den Fraktionen und deren Beziehungen entsprechen. 
Der Graph kann demzufolge analysiert und dargestellt werden, um unterschiedliche Muster in den Interaktionen zwischen zwei Fraktionen zu finden bzw. zu visualisieren.
Diese Arbeit beschäftigte sich mit der Erstellung einer Applikation, die den vorgenannten Graph in automatisierter Art und Weise aus dem schon vorgestellten Kommunikationsmodell ableiten und in einer Graph-Datenbank speichern kann.\newline 
Da das bereitgestellte Kommunikationsmodell unterschiedliche Typen von Beziehungen\footnote{Person $\rightarrow$ Person, Fraktion $\rightarrow$ Person und Person $\rightarrow$ Fraktion} enthält, war ein mengentheoretisches Verfahren zur Aggregation der Daten notwendig. Das Verfahren ist in Kapitel 2 näher beschrieben. 
Kapitel 3 enthält die funktionale bzw. nicht-funktionale Anforderungen der Applikation und Kapitel 4 beschreibt deren Umsetzung und die dazu genutzten Technologien. 
Kapitel 5 fasst die Ergebnisse zusammen. Weiterhin werden Möglichkeiten zur Optimierung der Applikation betrachtet. 
