\section{Anforderungsanalyse und Konzept}\label{sec:02_03_anforderungen_konzept}

\subsection{Organe der EU}

\subsubsection{Das europäische Parlament}\label{subsubsec:EuroParlament}

Das Europäische Parlament ist das Gesetzgebungsorgan der EU. Es wird alle fünf Jahre direkt von den Bürgerinnen und Bürgern der EU gewählt. Die letzten Wahlen fanden im Mai 2019 statt. Die wichtigsten Informationen über das europäische Parlament sind hier zusammengefasst:
\begin{itemize}
  \item Rolle: Direkt gewähltes EU-Organ mit Zuständigkeit für Gesetzgebung, Aufsicht und Haushalt
  \item Mitglieder: 705 Mitglieder des Europäischen Parlaments
  \item Präsident: David-Maria Sassoli
  \item Gegründet: 1952 als Gemeinsamen Versammlung der Europäischen Gemeinschaft für Kohle und Stahl, 1962 als Europäisches Parlament, erste direkte Wahl 1979
    \item Standort: Straßburg (Frankreich), Brüssel (Belgien), Luxemburg
    \item Webseite: \url{https://www.europarl.europa.eu/portal/de}
    \item Das Logo des europäischen Parlaments ist auf die Abbildung~\ref{fig:logoEuroParlament} zu sehen
\end{itemize}

\begin{figure}[H]
\centering
    \includegraphics[width=.8\textwidth]{images/Euro_Parlament.png}
    \caption{Das Logo des europäischen Parlaments}
    \label{fig:logoEuroParlament}
\end{figure}

\subsubsection{Der europäische Rat}\label{subsubsec:EuroRat}

Im Europäischen Rat kommen die Staats- und Regierungschefs der EU-Länder zusammen, um die politische Agenda der EU festzulegen. Er ist die höchste Ebene der politischen Zusammenarbeit zwischen den EU-Ländern.\newline
Als eines der sieben amtlichen Organe der EU tritt der Rat unter einem ständigen Vorsitz auf (zumeist vierteljährlichen) Tagungen der EU-Spitzen zusammen.\newline
Die wichtigsten Informationen über den europäischen Rat sind hier zusammengefasst:
\begin{itemize}
  \item Rolle: Bestimmung der allgemeinen politischen Zielvorstellungen und Prioritäten der Europäischen Union
  \item Mitglieder: Staats- und Regierungschefs der EU-Länder, Präsident des Europäischen Rates, Präsident der Europäischen Kommission
  \item Vorsitz: Charles Michel
  \item Gegründet: 1974 (informelles Forum), 1992 (offizieller Status), 2009 (offizielles EU-Organ)
    \item Standort: Brüssel, Belgien
    \item Webseite: \url{https://www.consilium.europa.eu/de/}
    \item Das Logo des europäischen Rats ist auf die Abbildung~\ref{fig:logoEuroRat} zu sehen
\end{itemize}

\begin{figure}[H]
\centering
    \includegraphics[width=.5\textwidth]{images/Euro_Rat.png}
    \caption{Das Logo des europäischen Rats}
    \label{fig:logoEuroRat}
\end{figure}

\subsubsection{Der Rat}\label{subsubsec:Rat}

Im Rat kommen Minister aus allen EU-Ländern zusammen, um Rechtsvorschriften zu diskutieren, zu ändern und anzunehmen. Außerdem koordinieren sie ihre Politikbereiche. Alle auf den Ratstagungen anwesenden Minister sind befugt, „für die Regierungen der von ihnen vertretenen Mitgliedstaaten verbindlich zu handeln“.\newline
Zusammen mit dem Europäischen Parlament ist der Rat der Europäischen Union das Hauptbeschlussorgan der EU.\newline
Nicht zu verwechseln mit dem europäischer Rat~\ref{subsubsec:EuroRat} und dem Europarat, der keine Einrichtung der EU ist.\newline
Die wichtigsten Informationen über den Rat sind hier zusammengefasst:
\begin{itemize}
  \item Rolle: Stimme der Regierungen von EU-Mitgliedsländern, die Gesetze annehmen und die EU-Politik koordinieren.
  \item Mitglieder: Minister aus jedem EU-Land, je nach behandeltem Politikbereich
  \item Vorsitz: Jedes EU-Land übernimmt wechselweise den Ratsvorsitz für sechs Monate
  \item Gegründet: 1958 (als Rat der Europäischen Wirtschaftsgemeinschaft)
    \item Sitz: Brüssel, Belgien
    \item Webseite: \url{https://www.consilium.europa.eu/de/}
    \item Das Logo des europäischen Rats ist auf die Abbildung~\ref{fig:logoEuroRat} zu sehen
\end{itemize}

\subsubsection{Die Europäische Kommission}\label{subsubsec:EuroKommission}

Die europäische Kommission ist die politisch unabhängige Exekutive der EU. Sie ist allein zuständig für die Erarbeitung von Vorschlägen für neue europäische Rechtsvorschriften und setzt die Beschlüsse des Europäischen Parlaments und des Rates der EU um.\newline
Die wichtigsten Informationen über die europäische Kommission sind hier zusammengefasst:
\begin{itemize}
  \item Rolle: Fördert die allgemeinen Interessen der EU durch Vorschläge für neue europäische Rechtsvorschriften und deren Durchsetzung. Setzt Strategien um und verwaltet den EU-Haushalt.
  \item Zusammensetzung: Ein Kommissionsmitglied aus jedem EU-Land. Gemeinsam bildet dieses Team das „Kollegium“.
  \item Präsidentin: Ursula von der Leyen
  \item Gegründet: 1958
    \item Sitz: Brüssel, Belgien
    \item Webseite: \url{https://ec.europa.eu/info/index_de}
    \item Das Logo des europäischen Rats ist auf die Abbildung~\ref{fig:logoEuroKomission} zu sehen
\end{itemize}

\begin{figure}[H]
\centering
    \includegraphics[width=.5\textwidth]{images/Euro_Komission.png}
    \caption{Das Logo der europäischen Komission}
    \label{fig:logoEuroKomission}
\end{figure}

\subsection{Einrichtungen der EU}

\subsubsection{Gerichtshof der Europäischen Union (EuGH)}\label{subsubsec:EuroGerichtshof}

Der Gerichtshof der Europäischen Union legt das EU-Recht aus und gewährleistet damit, dass es in allen EU-Ländern auf die gleiche Weise angewendet wird. Außerdem entscheidet er in Rechtsstreitigkeiten zwischen nationalen Regierungen und EU-Institutionen.\newline
In bestimmten Fällen können Privatpersonen, Unternehmen oder Organisationen ihn in einer Streitsache mit einer EU-Institution einschalten, wenn diese ihrer Auffassung nach ihre Rechte verletzt hat.\newline
Die wichtigsten Informationen über den Gerichtshof der Europäischen Union (EuGH) sind hier zusammengefasst:
\begin{itemize}
  \item Aufgaben: Gewährleisten, dass EU-Recht in allen EU-Mitgliedsländern auf die gleiche Weise angewendet wird und dafür sorgen, dass Länder und EU-Institutionen das EU-Recht einhalten.
  \item Mitglieder:
  \begin{itemize}
    \item Gerichtshof: Ein/-e Richter/-in aus jedem EU-Land, dazu elf Generalanwälte/-anwältinnen
    \item Gericht: zwei Richter aus jedem EU-Land
  \end{itemize}
  \item Gründung: 1952
  \item Ort: Luxemburg
  \item Sitz: Brüssel, Belgien
  \item Webseite: \url{https://curia.europa.eu/jcms/jcms/j_6/de/}
  \item Das Logo des europäischen Rats ist auf die Abbildung~\ref{fig:logoEuroGerichtshof} zu sehen
\end{itemize}

\begin{figure}[H]
\centering
    \includegraphics[width=.5\textwidth]{images/Euro_Gerichtshof.png}
    \caption{Das Logo des Gerichtshofs der Europäischen Union}
    \label{fig:logoEuroGerichtshof}
\end{figure}

\subsubsection{Europäische Zentralbank (EZB) }\label{subsubsec:EuroZentralBank}

Die Europäische Zentralbank (EZB) verwaltet den Euro und ist für die Gestaltung und Durchführung der Wirtschafts- und WährungspolitikDiesen Link in einer anderen Sprache aufrufen zuständig. Ihr wichtigstes Ziel ist die Preisstabilität, mit der das Wirtschaftswachstum und die Schaffung von Arbeitsplätzen unterstützt werden.\newline
Die wichtigsten Informationen über den Gerichtshof der Europäischen Union (EuGH) sind hier zusammengefasst:
\begin{itemize}
  \item Rolle: Verwaltung des Euro, Gewährleistung der Preisstabilität und Umsetzung der Wirtschafts- und Währungspolitik der EU
  \item Präsidentin: Christine Lagarde
  \item Mitglieder: Präsident/in und Vizepräsident/in der EZB sowie die Präsidenten der nationalen Zentralbanken aller EU-Mitgliedstaaten
  \item Gründung: 1998
  \item Standort: Frankfurt (Deutschland)
  \item Webseite: \url{https://www.ecb.europa.eu/home/html/index.en.html}
  \item Das Logo der europäischen Zentralbank ist auf die Abbildung~\ref{fig:logoEuroZentralBank} zu sehen
\end{itemize}

\begin{figure}[H]
\centering
    \includegraphics[width=.9\textwidth]{images/Euro_Zentralbank.png}
    \caption{Das Logo der Europäische Zentralbank}
    \label{fig:logoEuroZentralBank}
\end{figure}

\subsubsection{Weitere Einrichtungen der EU}\label{subsubsec:weitereEinrichtungen}

Hier sind noch weitere Einrichtungen der europäischen Union aufgelistet:
\begin{itemize}
  \item Europäischer Rechnungshof mit Sitz in Luxemburg
  \begin{itemize}
    \item Rolle: Kontrolle der ordnungsgemäßen Erhebung und Verwendung der EU-Mittel und Beitrag zur Verbesserung des Finanzmanagements der EU
    \item Präsident: Klaus-Heiner Lehne
    \item Mitglieder: 1 aus jedem EU-Land
  \end{itemize}
  \item Europäischer Auswärtiger Dienst (EAD) mit Sitz in Brüssel, Belgien
  \begin{itemize}
    \item Rolle: Der EAD pflegt die diplomatischen Beziehungen der EU zur übrigen Welt und setzt die Außen- und Sicherheitspolitik der EU um.
    \item Hoher Vertreter der EU für Außen- und Sicherheitspolitik: Josep Borrell
    \item Gegründet: 2011
  \end{itemize}
  \item Europäischer Wirtschafts- und Sozialausschuss (EWSA) mit Sitz in Brüssel, Belgien
  \begin{itemize}
    \item Rolle: Beratende Einrichtung, die Arbeitgeber- und Arbeitnehmerorganisationen sowie andere Interessengruppen vertritt
    \item Präsident: Luca Jahier, Mitglieder: 326 aus allen EU-Ländern
    \item Gegründet: 1957
  \end{itemize}
  \item Europäischer Ausschuss der Regionen (AdR) mit Sitz in Brüssel, Belgien
  \begin{itemize}
    \item Aufgabe: Beratende Einrichtung, die die regionalen und lokalen Gebietskörperschaften Europas vertritt
    \item Präsident: Apostolos Tzitzikostas
    \item Gegründet: 1994
  \end{itemize}
  \item und fünf weitere, siehe~\cite{euroInstitution} für mehr Informationen
\end{itemize}